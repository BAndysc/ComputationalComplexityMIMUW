\documentclass[landscape, 9pt, twocolumn]{article}
\usepackage[utf8]{inputenc}

\usepackage[margin=0.5in,top=0.7in,footskip=.25in]{geometry}

\setlength{\parindent}{0em}
\setlength{\parskip}{0.4em}
\usepackage{dirtytalk}
\usepackage{amsmath}
\usepackage{amssymb}
\usepackage{amsfonts}
\usepackage{minted}
\usepackage{tikz}
\usepackage{amsmath,amsthm}
\usepackage{bm}
\usepackage[normalem]{ulem}


\newcommand{\maysubsetneq}{\stackrel{?}{\subsetneq}}
\newcommand{\maysubseteq}{\stackrel{?}{\subseteq}}
\newcommand{\mayeq}{\stackrel{?}{=}}
\newcommand{\mayneq}{\stackrel{?}{\neq}}
\newcommand{\ra}{\rightarrow}
\newcommand{\la}{\leftarrow}
\newcommand{\PSPACE}{\textbf{PSPACE}}
\newcommand{\NSPACE}{\textbf{NSPACE}}
\newcommand{\EXPSPACE}{\textbf{EXPSPACE}}
\newcommand{\NEXPSPACE}{\textbf{NEXPSPACE}}
\newcommand{\NPSPACE}{\textbf{NPSPACE}}
\newcommand{\ZPP}{\textbf{ZPP}}
\newcommand{\BPP}{\textbf{BPP}}
\renewcommand{\P}{\textbf{P}}
\newcommand{\NP}{\textbf{NP}}
\renewcommand{\L}{\textbf{L}}
\newcommand{\NL}{\textbf{NL}}
\newcommand{\AL}{\textbf{AL}}
\newcommand{\PH}{\textbf{PH}}
\newcommand{\AP}{\textbf{AP}}
\newcommand{\BQP}{\textbf{BQP}}
\newcommand{\IP}{\textbf{IP}}
\renewcommand{\AP}{\textbf{AP}}
\newcommand{\RP}{\textbf{RP}}
\newcommand{\PP}{\textbf{PP}}
\newcommand{\AC}{\textbf{AC}}
\newcommand{\NC}{\textbf{NC}}
\newcommand{\EXPTIME}{\textbf{EXPTIME}}
\newcommand{\NEXPTIME}{\textbf{NEXPTIME}}
\newcommand{\DTIME}{\textbf{DTIME}}
\newcommand{\DSPACE}{\textbf{DSPACE}}
\newcommand{\NTIME}{\textbf{NTIME}}
\newcommand{\co}{\textbf{co}}
\renewcommand{\u}{\textbf{u-}}
\newcommand{\poly}{\textbf{/poly}}
\newcommand{\polyL}{\textbf{polyL}}

\usetikzlibrary{chains,fit,shapes,shapes.arrows}
\title{ZŁO DO EGZAMINU}
\author{}
\date{}

\begin{document}

\makeatletter
\AtBeginEnvironment{minted}{\dontdofcolorbox}
\def\dontdofcolorbox{\renewcommand\fcolorbox[4][]{##4}}
\makeatother

\section{Znane zależności}



\tikzset{every picture/.style={line width=0.75pt}} %set default line width to 0.75pt        

\begin{tikzpicture}[x=0.75pt,y=0.75pt,yscale=-1,xscale=1]
%uncomment if require: 
\path (0,647); %set diagram left start at 0, and has height of 647


% Text Node
\draw (167,310) node  [align=left] {$\displaystyle \ZPP^{\ZPP} =\co\ZPP\ =\ \ZPP\ =\ \co\RP\ \cap \RP$};
% Text Node
\draw (226,350) node  [align=left] {$\displaystyle \RP$};
% Text Node
\draw (119,350) node  [align=left] {$\displaystyle \co\RP$};
% Text Node
\draw (145.4,334.07) node [rotate=-323.84] [align=left] {$\displaystyle \supseteq $};
% Text Node
\draw (205.26,333.74) node [rotate=-45.04] [align=left] {$\displaystyle \subseteq $};
% Text Node
\draw (257,500) node  [align=left] {$\displaystyle \subseteq $};
% Text Node
\draw (178,390) node  [align=left] {$\displaystyle \BPP=\co\BPP$};
% Text Node
\draw (112.74,426.26) node [rotate=-45.04] [align=left] {$\displaystyle \subseteq $};
% Text Node
\draw (334.5,209) node  [align=left] {$\displaystyle \PP$};
% Text Node
\draw (281,400) node  [align=left] {$\displaystyle \NP\ =\ \textbf{d}\IP$};
% Text Node
\draw (179,413) node [rotate=-90] [align=left] {$\displaystyle \subseteq $};
% Text Node
\draw (80,400) node  [align=left] {$\displaystyle \co\NP$};
% Text Node
\draw (142.74,366.26) node [rotate=-45.04] [align=left] {$\displaystyle \subseteq $};
% Text Node
\draw (203.4,366.6) node [rotate=-323.84] [align=left] {$\displaystyle \supseteq $};
% Text Node
\draw (95.4,374.26) node [rotate=-323.84] [align=left] {$\displaystyle \supseteq $};
% Text Node
\draw (242.74,376.26) node [rotate=-45.04] [align=left] {$\displaystyle \subseteq $};
% Text Node
\draw (178,442.53) node  [align=left] {$\displaystyle \PP\ =\ \co\PP$};
% Text Node
\draw (245.4,424.26) node [rotate=-323.84] [align=left] {$\displaystyle \supseteq $};
% Text Node
\draw (179,473) node [rotate=-90] [align=left] {$\displaystyle \subseteq $};
% Text Node
\draw (137,510) node  [align=left] {$\displaystyle  \begin{array}{l}
\co\PSPACE=\PSPACE\ =\\
\ \co\NPSPACE=\NPSPACE\ =\\
\AP\ =\ \IP
\end{array}$};
% Text Node
\draw (312.5,500) node  [align=left] {$\displaystyle \EXPTIME$};
% Text Node
\draw (369.5,500) node  [align=left] {$\displaystyle \subseteq $};
% Text Node
\draw (422.5,500) node  [align=left] {$\displaystyle \NEXPTIME$};
% Text Node
\draw (426.4,534.26) node [rotate=-323.84] [align=left] {$\displaystyle \supseteq $};
% Text Node
\draw (315.5,570) node  [align=left] {$\displaystyle  \begin{array}{l}
\EXPSPACE\ =\ \NEXPSPACE\ =\\
\co\NEXPSPACE\ =\ \co\EXPSPACE
\end{array}$};
% Text Node
\draw (185.5,270) node  [align=left] {$\displaystyle \P\ =\ \AL$};
% Text Node
\draw (179,293) node [rotate=-90] [align=left] {$\displaystyle \subseteq $};
% Text Node
\draw (185,140) node  [align=left] {$\displaystyle \NL=\ \co\NL$};
% Text Node
\draw (179,163) node [rotate=-90] [align=left] {$\displaystyle \subseteq $};
% Text Node
\draw (176.5,100) node  [align=left] {$\displaystyle \L$};
% Text Node
\draw (180,123) node [rotate=-90] [align=left] {$\displaystyle \subseteq $};
% Text Node
\draw (200,100) node  [align=left] {$\displaystyle \subseteq $};
% Text Node
\draw (235.5,100) node  [align=left] {$\displaystyle \textbf{polyL}$};
% Text Node
\draw (343,173) node  [align=left] {$\displaystyle \BPP=\co\BPP$};
% Text Node
\draw (391.5,209) node  [align=left] {$\displaystyle \P\poly$};
% Text Node
\draw (377.74,195.26) node [rotate=-45.04] [align=left] {$\displaystyle \subseteq $};
% Text Node
\draw (297.6,194.74) node [rotate=-323.84] [align=left] {$\displaystyle \supseteq $};
% Text Node
\draw (273,209) node  [align=left] {$\displaystyle \BQP$};
% Text Node
\draw (338,197) node [rotate=-90] [align=left] {$\displaystyle \subseteq $};
% Text Node
\draw (268,227) node [rotate=-90] [align=left] {$\displaystyle \subseteq $};
% Text Node
\draw (273,239) node  [align=left] {$\displaystyle \PSPACE...$};
% Text Node
\draw (176.5,50) node  [align=left] {$\displaystyle \u\NC^{1}$};
% Text Node
\draw (180,73) node [rotate=-90] [align=left] {$\displaystyle \subseteq $};
% Text Node
\draw (176.5,180) node  [align=left] {$\displaystyle \u\AC^{1}$};
% Text Node
\draw (180,203) node [rotate=-90] [align=left] {$\displaystyle \subseteq $};
% Text Node
\draw (176,225.5) node  [align=left] {$\displaystyle \u\NC^{2}\cdots$};
% Text Node
\draw (179.5,248.5) node [rotate=-90] [align=left] {$\displaystyle \subseteq $};
% Text Node
\draw (176.5,10) node  [align=left] {$\displaystyle \u\AC^{0}$};
% Text Node
\draw (180,33) node [rotate=-90] [align=left] {$\displaystyle \subsetneq $};
% Text Node
\draw (55.5,51) node [color={rgb, 255:red, 155; green, 155; blue, 155 }  ,opacity=1 ] [align=left] {języki regularne};
% Text Node
\draw (56,268) node [color={rgb, 255:red, 155; green, 155; blue, 155 }  ,opacity=1 ] [align=left] {języki \\bezkontekstowe};
% Text Node
\draw (137,50) node  [align=left] {$\displaystyle \subseteq $};
% Text Node
\draw (127,270) node  [align=left] {$\displaystyle \subseteq $};


\end{tikzpicture}




%
%\tikzset{every picture/.style={line width=0.75pt}} %set default line width to 0.75pt        
%
%\begin{tikzpicture}[x=0.75pt,y=0.75pt,yscale=-1,xscale=1]
%%uncomment if require: \path (0,188); %set diagram left start at 0, and has height of 188
%
%
%% Text Node
%\draw (34,41) node  [align=left] {$\displaystyle \P\subseteq \ZPP$};
%% Text Node
%\draw (123,10) node  [align=left] {$\displaystyle \RP$};
%% Text Node
%\draw (120,80) node  [align=left] {$\displaystyle \co\RP$};
%% Text Node
%\draw (83.6,24.07) node [rotate=-321.34] [align=left] {$\displaystyle \subseteq $};
%% Text Node
%\draw (90.74,63.26) node [rotate=-45.04] [align=left] {$\displaystyle \subseteq $};
%% Text Node
%\draw (160.74,41.26) node [rotate=-45.04] [align=left] {$\displaystyle \subseteq $};
%% Text Node
%\draw (197.5,70) node  [align=left] {$\displaystyle \BPP$};
%% Text Node
%\draw (240.74,16.26) node [rotate=-45.04] [align=left] {$\displaystyle \subseteq $};
%% Text Node
%\draw (240.6,54.07) node [rotate=-321.34] [align=left] {$\displaystyle \subseteq $};
%% Text Node
%\draw (267.5,36) node  [align=left] {$\displaystyle \PP$};
%% Text Node
%\draw (223.26,83.74) node [rotate=-45.04] [align=left] {$\displaystyle \subseteq $};
%% Text Node
%\draw (267.5,100) node  [align=left] {$\displaystyle \P\poly$};
%% Text Node
%\draw (156.4,79.93) node [rotate=-321.34] [align=left] {$\displaystyle \subseteq $};
%% Text Node
%\draw (203,10) node  [align=left] {$\displaystyle \NP$};
%% Text Node
%\draw (165,11) node  [align=left] {$\displaystyle \subseteq $};
%% Text Node
%\draw (196.5,39) node  [align=left] {?};
%% Connection
%\draw [color={rgb, 255:red, 165; green, 165; blue, 165 }  ,draw opacity=1 ]   (201.72,23.99) -- (198.78,56.01) ;
%\draw [shift={(198.6,58)}, rotate = 275.24] [color={rgb, 255:red, 165; green, 165; blue, 165 }  ,draw opacity=1 ][line width=0.75]    (10.93,-3.29) .. controls (6.95,-1.4) and (3.31,-0.3) .. (0,0) .. controls (3.31,0.3) and (6.95,1.4) .. (10.93,3.29)   ;
%\draw [shift={(201.9,22)}, rotate = 95.24] [color={rgb, 255:red, 165; green, 165; blue, 165 }  ,draw opacity=1 ][line width=0.75]    (10.93,-3.29) .. controls (6.95,-1.4) and (3.31,-0.3) .. (0,0) .. controls (3.31,0.3) and (6.95,1.4) .. (10.93,3.29)   ;
%
%\end{tikzpicture}
%
%

\subsection{Podstawowe klasy}
\begin{itemize}
    %\item $\P \subseteq \NP \subseteq \bigcup_{c > 1} \DTIME(2^{n^c})$ %
    \item $\DSPACE(f(n)) \subseteq  \bigcup_{c >0} \DTIME(n * c^{f(n)})$ 
    \item if $f(n) \geq log(n)$, then $\DSPACE(f(n)) \subseteq  \bigcup_{c >0} \DTIME(c^{f(n)})$
    %\item $\P \subsetneq \P\poly$ ($\P\poly \not\subseteq \P$)
    %\item $\NPSPACE = \PSPACE = \co\NPSPACE$, zamknięte na przecięcie %todo jest na rysunku
   
    \item $\NTIME(f(n)) \subseteq \DSPACE(f(n))$
    \item $\NSPACE(f(n)) \subseteq \cup_{c \in \mathbb{N}} \DTIME(c^{f(n) + log(n)})$
    \item (Savitch's theorem) if $f(n)=\Omega(log n)$, then $\NSPACE(f(n)) \subseteq \DSPACE(f(n)^2)$
%    \item $\NL = \co\NL$
    \item if $S(n) \geq log(n)$, then $\NSPACE(S(n))= \co\NSPACE(S(n))$
     %\item $\L \subsetneq \PSPACE$ %todo oczywiste ze space hierarchy 
    \item $\textbf{polyL} \neq \PSPACE$
    \item $\textbf{polyL} \neq \P$

\end{itemize}
\subsection{Obwody}
\begin{itemize}
    \item $\AC = \NC$, $\u\AC = \u\NC$
    \item $\AC^k \subseteq \NC^{k+1}$, $\u\AC^k \subseteq \u\NC^{k+1}$ 
%    \item $\u\NC^1 \subseteq \L \subseteq \NL \subseteq \u\AC^1 \subseteq \u\NC^2$
%    \item języki regularne $\subseteq \u\NC^1$ (ale nie $\AC^0$)
    \item $\u\AC \subseteq \P$
    \item $\u\AC^0 \neq \u\NC^1$
    \item $\u\NC \neq \PSPACE$
\end{itemize}
\subsection{Klasy probabilistyczne}
\begin{itemize}
%    \item $\P \subseteq \RP \subseteq \NP$
%    \item $\NP \subseteq \PP$
%    \item $\RP \subseteq \BPP \subseteq PP$
    \item if $\NP \subseteq \BPP$, then $\NP \subseteq \RP$
%    \item $\ZPP = \RP \cap \co\RP$
%    \item $\BPP \subseteq P\poly$
%    \item $\BPP \subseteq \PSPACE$
%    \item $\ZPP \subseteq \BPP$
%    \item $\co\RP \subseteq \BPP$
%    \item $\ZPP = \co\ZPP$
%    \item $\BPP \subseteq \textbf{BQP}  \subseteq \PSPACE$
\end{itemize}


\subsection{Polynomial hierarchy i alternating machine}
\begin{itemize}
    \item $\sum_k^p \subseteq \sum_{k+1}^p$, $\sum_k^p \subseteq \prod_{k+1}^p$, $\prod_k^p \subseteq \sum_{k+1}^p$, $\prod_k^p \subseteq \prod_{k+1}^p$
    \item if $\sum_k^p = \prod_k^p$, then $\sum_k^p = \PH$
    \item $\PH \subseteq \PSPACE$
%    \item $\AL = \P$
%    \item $\AP = \PSPACE$
    
    
    

   
    
\end{itemize}

\subsection{PCP i IP}
\begin{itemize}
    \item $PCP(poly(n), 0) = \co\RP$
    \item $PCP(0, poly(n))= \NP$
    \item $PCP(log(n), poly(n)) = \NP$
    \item $PCP(log(n), 1) = \NP$
    \item $PCP(poly(n), 1) = \EXPTIME$
    \item $PCP(1,1) = \P $
%    \item $\textbf{d}\IP = \NP$
%    \item $\IP = \PSPACE$
\end{itemize}

\section{Nieznane zależności}
\begin{itemize}
    \item $\P \mayeq \NP$
    \item $\NL \mayeq \NP$
    \item $\NP \stackrel{?}{\not\subseteq}  \P\poly$
    \item $\u\NC \mayneq \P$
    \item $\NP \mayneq \co\NP$ (rownoważnie: $\NP^\NP \mayeq \NP$)
    \item $\NP \cap \co\NP \mayeq \P$
    \item $\L \mayeq \NL$
    \item $\L \mayeq \NP$
    \item $\L \mayeq \P$
    \item $\P \mayeq \PSPACE$
    \item $\NP \mayneq \PSPACE$
    \item $\NC \mayeq \P$
    \item $\P \mayeq \RP$
    \item $\BPP \mayeq \P$
    \item $\BPP \maysubseteq \ZPP$
    
\end{itemize}

\section{Definicje}
\begin{enumerate}
    \item Time-constructible $f(n)$
    
    It's function $f(n)$ for which there exists a machine M, which for input $1^n$
    \begin{itemize}
        \item outputs a word of length precisely $f(n)$
        \item works in time $O(f(n))$
    \end{itemize}
    If f and g are time-constructible, then $f+g$, $f*g$, $f^g$ as well. Functions n, $\lfloor n*log(n) \rfloor$, $n^k$, $k^n$ are time-constructible. But $\lfloor log(n) \rfloor$ is NOT.
    
    \item Space-constructible $f(n)$
    
    It's function $f(n)$ for which there exists a machine M, which for input $1^n$
    \begin{itemize}
        \item outputs a word of length precisely $f(n)$
        \item works in space $O(f(n))$
    \end{itemize}
    Every time-constructible function is also space-constructible.
    If f and g are space-constructible, then $f+g$, $f*g$, $f^g$ as well. Functions n, $\lfloor log(n) \rfloor$, $n^k$, $k^n$ are space-constructible. But $\lfloor log(log(n+2)) \rfloor$ is NOT.
    
    \item P/poly class
    
    Languages recognizable in polynomial time by a machine with advice (of polynomial size)
    
    \item Uniform sequences of circuits
    
    A sequence of circuit $C_0, C_1, C_2,...$ is uniform if it is computable in logarithmic space. (There exists a TM working in logarithmic space, which on input $1^n$ outputs the representation of circuits $C_n$.
    
    \item Turing reductions / Cook reductions
    
    A language L is Turing-reducible to K if there exists a machine with an oracle for K, which recognizes L.
    
    By limiting the resources of M, one can talk about polynomial-time Turing reductions (often called Cook reductions), logarithmic-space Turing reductions, etc.
    
    
    \item Karp reductions
    
    Idea: we can make only a single query to the language K and we cannot negate the answer.
    
    A language $L \subseteq \sum^*$ is Karp-reducible to $K \subseteq \Gamma^*$ if there exists a function $f:\sum^* \rightarrow \Gamma^*$ computable in logarithmic space (sometimes in polynomial time), such that $w \in L \iff f(w) \in K$ for every word $w \in \sum^*$
    
    \item Levin reductions
    
    The idea of Levin reductions: additionally a witness for the first problem allows to recover a witness for the second problem.
    
    It is a reduction between relations $R_1, R_2 \subseteq \sum^* \times \sum^*$
    
    $R_1$ is Levin-reducible to $R_2$ if there are functions $f:\sum^* \rightarrow \sum^*$, $g,h:\sum^* \times \sum^* \rightarrow \sum^*$  (computable in logarithmic space / polynomial time) such that:
    
    $R_1(x,y) \Rightarrow R_2(f(x), g(x,y))$
    
    $R_2(f(x),z) \Rightarrow R_1(x, h(x,z))$ (for all $x,y,z \in \sum^*$)
    
    \item Polynomial hierarchy
    \begin{itemize}
        \item $PH = \cup_k \sum_k^p$
        \item $\sum_1^p= \NP$
        \item $\prod_1^p= \co\NP$
        \item $\sum_2^p= \NP^{\NP}$
        \item If there exists a \PH-complete language, then $\PH = \sum_k^p$ for some k
        \item If the classes $\sum_k^p$ are all different, then $\PH \neq \PSPACE$
    \end{itemize}
    
    \item class $\polyL$
    
     $\polyL = \DSPACE((log (n))^{O(1)})$
    
    \item class $\RP$
    
    A language $L$ is in $\RP$ iff there is a machine M with source of random bits, working in polynomial time and such that:
    \begin{itemize}
        \item $w \in L \Rightarrow Pr_s[(w,s) \in L_M] \geq 0.5$
        \item $w \notin L \Rightarrow \nexists s. (w,s) \in L_M$
    \end{itemize}
    
    \item class $\PP$
    
    Like $\RP$, but:
    \begin{itemize}
        \item $w \in L \Rightarrow Pr_s[(w,s) \in L_M] \geq 0.5$
        \item $w \notin L \Rightarrow Pr_s[(w,s) \in L_M] < 0.5$
    \end{itemize}
    
    \item class $\BPP$
    
    Like $\RP$, but:
    \begin{itemize}
        \item $w \in L \Rightarrow Pr_s[(w,s) \in L_M] \geq 3/4$
        \item $w \notin L \Rightarrow Pr_s[(w,s) \in L_M] \leq 1/4$
    \end{itemize}
    
    \item Fixes-parameter tractability
    \begin{itemize}
        \item a parameter - a function from input words to natural numbers
        \item a problem is FPT with respect to a parameter k, if it has complexity $f(k)*n^c$ (important: the exponent does not depend on k)
    \end{itemize}
    
    \item Approximation
    
    $\rho$-approximation algorithm returns a solution that: 
    
    $\rightarrow$ for maximization problems: $solution \geq optimum*\rho$
    
    $\rightarrow$ for minimization problems: $solution \leq optimum*\rho$
    
    Different possibilities: 
    \begin{itemize}
        \item approximation impossible within any constant factor (unless p=NP)
        \item there exists an approximation algorithm with a constant factor
        \item for every $\epsilon > 0$ there is an $(1+\epsilon)$-approximation algorithm (minimization problems) or $(1-\epsilon)$-approximation algorithm (maximization problems), we say that there exists a PTAS (polynomial time approximation scheme);
        
        if the exponent in the algoritm does not depend on $\epsilon$, we say that there exists a strong PTAS 
    \end{itemize}
    
    Examples:
    \begin{itemize}
        \item Traveling salesmen problem: no approximation with constant factor, unless $\P = \NP$.
        \item MAX-CLIQUE – approximation almost impossible
        \item MAX-3CNFSAT – 7/8-approximation possible,
        better approximation impossible
        \item There is a 2-approximation of VERTEX-COVER
        \item knapsack problem – there is a strong PTAS – for every $\epsilon > 0 $ there is a $(1-\epsilon)$-approximation algorithm working in time $O(n^3)$ 
    \end{itemize}
    
    
    

    
\end{enumerate}

\section{Twierdzenia}
\begin{enumerate}
    \item Theorem (linear speed-up)
    
    If a language L is decidable in time T(n), then for every constant $c > 0$ it is also decidable in time $c*T(n) + O(n)$
    
    \item Sipser's theorem
    
    Consider a machine M working in space $S(n)$, but not necessarily having the halting property. Then there exists a machine $M'$ such that:
    \begin{itemize}
        \item $L(M') = L(M)$
        \item $M'$ works in space $S(n)$
        \item $M'$ halts on every input
    \end{itemize}
    Corrollary: If a language L is semidecidable, but not decidable, then every machine M recognizing L on some word w uses infinite memory.
    
    \item Universal machine's theorem
    
    There exists a universal Turing Machine U (an "interpreter"), such that $U(\langle M \rangle,w) = M(w)$. If M works in time T(w) and space S(w), then U works in time  $O(T|w|)*log(T|w|)))$ and space $O(S|w|)).$
    
    \item Space hierarchy theorem
    
    If function $g(n)$ is space-constructible and $f(n)=o (g(n))$ then $\DSPACE(f(n)) \neq \DSPACE(g(n))$
    
    \item Time hierarchy theorem
    
    If function $g(n)$ is time-constructible and $f(n)=o (g(n))$ then $\DTIME(f(n)) \neq \DTIME(g(n)log(n)))$
    
    \item Gap theorems
    
    There is a computable function $f(n) \geq n$ such that $\DTIME(f(n)) = \DTIME(2^{f(n))}$.
    
    There is a computable function $f(n)$ such that $\DSPACE(f(n)) = \DSPACE(2^{f(n))}$.
    
    \item Simulating machines by circuit
    
    Every language recognizable in time T(n) on a multi-tape machine can be recognized by a sequence of circuits $(C_{n})_{n \in \mathbb{N}}$ of depth $O(T(n))$ and $O(T(n)*log(T(n))))$. The circuit $C_n$ can be generated in logarithmic space.
    
    \item P/poly theorem
    
    A language belongs to P/poly iff it is recognizable by a sequence of circuits of polynomial size.
    
    \item Uniform sequences of circuits
    
    A language is recognizable by a uniform sequence of circuits iff it is in $\P$
    
    \item Ladner's theorem
    
    If $\P \neq \NP$, then there is a problem, which is in $\NP \backslash \P$, but is not  $\NP-hard$ with respect to polynomial-time reductions (so even more with respect to logarithmic-space reductions).
    
    
    \item Theorem (Berman 1978)
    
    If $\P \neq \NP$, then no language over a single-letter alphabet is $\NP-hard$ wrt. polynomial-time reductions (so even more wrt. logarithmic-space reductions)
    
   
    
    
    \item Theorem (Baker-Gill-Solovay, 1975)
    
     There exist languages $A$, $B$ such that $\P^A=\NP^A$ and $\P^B \neq \NP^B$

    \item Theorem (Courcelle 1990):
    Every property of graphs expressible in the MSO logic can be
    decided in time $f(s) * n$, where s is the treewidth.
    In this logic, we allow quantification over sets of nodes, and over sets of edges. For most properties, it is easy to express them in this logic.
    
    \item Amplification
    
    Let $L \in \RP$. Then, for every polynomial $q(n)$ there is a machine M with source of random bits, working in polynomial time and such that:
    \begin{itemize}
        \item $w \in L \Rightarrow Pr_s[(w,s) \in L_M] \geq 1-1/2^{q(n)}$
        \item $w \notin L \Rightarrow \nexists s. (w,s) \in L_M$
    \end{itemize}
\end{enumerate}

\section{Problemy Zupełne}
\begin{enumerate}
    \item $\NP$-completeness
    \begin{itemize}
        \item Hamiltonian path problem
        \item Traveling salesman problem
        \item Clique problem
        \item Knapsack problem
        \item Subgraph isomorphism problem
        \item Subset sum problem
        \item Vertex cover problem
        \item Independent set problem
        \item Dominating set problem
        \item Graph coloring problem
        \item SAT, 3-SAT
    \end{itemize}
    \item $\P$-completeness
    \begin{itemize}
        \item HORNSAT
        \item Context Free Grammar Membership - Given a context-free grammar and a string, can that string be generated by that grammar?
    \end{itemize}
    \item $\textbf{polyL}$-completeness
    \begin{itemize}
        \item NO complete problems
    \end{itemize}
    \item $\L$-completeness
    \begin{itemize}
        \item Almost every language in $\L$ is complete (except the empty language, and the language containing all words)
    \end{itemize}
    \item $\NL$-completeness
    \begin{itemize}
        \item 2-SAT
        \item Reachability in a directed graph
        \item Unreachability in a directed graph ($\in \co\NL = \NL$)
    \end{itemize}
    \item $\PSPACE$-completeness
    \begin{itemize}
        \item QBF
        \item Tiling problem
    \end{itemize}
\end{enumerate}

\section{Tricki}
\begin{itemize}
    \item maszyna ma wiele konfiguracji akceptujących, a chcielibyśmy mieć jedną?
    
    $\rightarrow$ dodajmy do maszyny stany, które po oryginalnym zaakceptowaniu jadą głowicami na początek taśmy, czyszczą wszystko po drodze i dopiero wtedy akceptują. I już - mamy jedną konfigurację akceptującą.
    
    \item zwykle redukujemy konkretny problem zupełny do naszego, ale nic nie stoi na przeszkodzie aby redukować dowolny język z danej klasy
    
%    \item $\\ \\$
%    \item $\\ \\$
%    \item $\\ \\$
%    \item $\\ \\$
%    \item $\\ \\$
%    \item $\\ \\$
\end{itemize}


\section{Egzamin 2014/2015}
\subsection{Problem 1}
Niech $R \subseteq \{0, 1\}^* \times \{0, 1\}^*$ będzie relacją na słowach nad alfabetem $\{0, 1\}$, spełniającą następujące warunki:
\begin{itemize}
    \item jeśli $(u, v) \in R$, to $u$ i $v$ mają tę samą długość,
    \item język $\{u\#v\ |\ (u, v) \in R\}$ jest w $\PSPACE$. 
\end{itemize}

Domknięcie przechodnie relacji $R$ definiujemy jako
\[
    R^+ = \{(u, v) : \exists_{n \in \mathbb{N}} \exists_{w_0, w_1, \cdots, w_n} u = w_0 \land v = w_n \land \forall_{0 \leq i \leq n} (w_i, w_{i+1}) \in R \}
\]

Udowodnij, że język $\{u\#v\ :\ (u, v) \in R^+\}$ jest w $\PSPACE$.

\subsubsection*{Rozwiązanie}

Ponieważ $\PSPACE = \NPSPACE$, to niedeterministycznie przechodzimy i już.

\subsection{Problem 2}
Załóżmy, że dla języka $L$ istnieje probabilistyczna maszyna Turinga $M$, która dla każdego słowa długości $n$ działa w czasie oczekiwanym $O(n)$ oraz niezależnie od wyników losowań akceptuje każde słowo z $L$ i odrzuca każde słowo spoza $L$. Udowodnij, że $L \in \DSPACE(n)$.

\subsubsection*{Rozwiązanie}

$w \in L \implies$ $M$ akceptuje $w$ (niezależnie od wyników losowań)

$w \not\in L \implies$ $M$ odrzuca $w$ (niezależnie od wyników losowań)

(zawsze działa dobrze, tylko czasami długo)

$\forall_{w}$ istnieje $r$ - ciąg bitów losowych $r$ powodujący, że $M$ na słowie $w$ zatrzyma się w czasie $\leq c \cdot n$ i $|r| \leq c \cdot n$


$\forall$ ciągów bitów losowych $r$ długości $= c \cdot n$ uruchamiamy $M$ na $w$ i na $r$

\sout{Chcemy pokazać, że $L \subseteq \DSPACE(n)$.
My twierdzimy, że $L = \ZPP(n)$
Wiadomo, że $\ZPP = \RP \cap \co\RP$, $\ZPP(n) = \RP(n) \cap \co\RP(n)$
$\RP(n) \subseteq \NP(n) = \NTIME(n) \subseteq \DSPACE(n)$}

\qed


\iffalse
\subsection{Problem 3}
Udowodnij, że następujący problem jest $\NP$-zupełny: dany alfabet $\Sigma$, niedeterministyczny automat skończony $A$ nad $\Sigma$ oraz liczba naturalna $n$ zapisana unarnie, czy istnieje słowo długości $n$ odrzucane przez automat $A$?
\fi

\section{Egzamin 2015/2016}
\subsection{Problem 1}

Opisz jak deterministycznie w pamięci logarytmicznej rozwiązać następujący problem. Dla danego ciągu wejściowego $w \in \{0, 1\}^∗$:
\begin{enumerate}
    \item sprawdź, czy $|w|$ jest postaci $n^2$ dla jakiejś liczby n,
    \item jeśli tak, to sprawdź czy $w$, rozumiane jako graf relacji binarnej na zbiorze $\{1, \cdots, n\}$, zadaje relację równoważności,
    \item jeśli tak, to wypisz (w reprezentacji binarnej) liczbę klas abstrakcji tej relacji.
\end{enumerate}

\subsubsection*{Rozwiązanie}

Intuicyjnie algorytmy w pamięci logarytmicznej korzystają ze stałej liczby wskaźników.

\begin{enumerate}
    \item Dodajemy kolejne liczby nieparzyste aż nie przekroczymy długości słowa, jak trafimy akurat w długość słowa to jest kwadratem
    
    \textbf{Prościej}: iterujemy po liczbach od $1$ do $n$, liczymy jej kwadrat (w pętli dodawanie) i sprawdzamy czy się zgadza
    \item \begin{minted}{python}
    zwrotna = True
    for x = 0 to n - 1:
        if not krawedz(x, x):
            zwrotna = False
    
    symetryczna = True
    for x = 0 to n - 1:
        for z = 0 to n - 1:
            if krawedz(x, z) and not krawedz(z, x):
                symetryczna = False
    
    przechodnia = True
    for x = 0 to n - 1:
        for y = 0 to n - 1:
            for z = 0 to n - 1:
                if krawedz(x, y) and krawedz(y, z):
                    if not krawedz(x, z):
                        przechodnia = False
                        
    rownowaznosci = zwrotna and przechodnia and symetryczna
    \end{minted}
    
    \item \begin{minted}{python}
    ileKlas = 1 # bo klasa z "0"
    for i = 0 to n - 1:
        czyNowaKlasaAbstrakcji = True
        for j = 0 to i - 1:
            if krawedz(i, j):
                czyNowaKlasaAbstrakcji = False
        if czyNowaKlasaAbstrakcji:
            ileKlas += 1
    \end{minted}
\end{enumerate}

\subsection{Problem 2}
Obwód logiczny o $2n$ bramkach wejściowych i jednej bramce wyjściowej definiuje graf skierowany o $2^n$ wierzchołkach w ten sposób, że jeśli na pewnej kombinacji wartości bramek wejściowych daje wartość $1$, to w grafie występuje krawędź z wierzchołka o numerze $K$ do wierzchołka o numerze $M$, gdzie $K$ jest binarnie reprezentowane przez ciąg wartości pierwszych $n$ bramek, a $M$ – drugich $n$ bramek.

Rozważmy następujący problem. Dane: obwód jak powyżej oraz dwie liczby binarne $K$, $M$ o $n$ bitach każda. Czy wierzchołek $M$ jest osiągalny z $K$ w grafie zdefiniowanym przez ten obwód?

Udowodnij, że ten problem jest $\PSPACE$-zupełny

\subsubsection*{Rozwiązanie}

\begin{itemize}
    \item jest $\PSPACE$
    
    niedeterministycznie zgadujemy ścieżkę między nimi i $n$ razy symulujemy ten obwód (Dla każdej pracy wierzchołków na tej ścieżce sprawdzamy czy istnieje między nimi krawędź. Zatem jest $\NPSPACE$, czyli jest $\PSPACE$. Ważne, że wygenerowany obwód jest wielomianowej wielkości.
    
    \item $\PSPACE$-trudne
\end{itemize}

\subsection{Problem 3}
Dla dowolnego języka $L \subseteq \Sigma^*$, i dla słowa $w= w_1w_2\cdots w_n \in \Sigma^n$, słowo $w^{(L)} \in \{0, 1\}^n$ jest zdefiniowane następująco:
\[
    \left(w^{(L)}\right)_i = \left\{ 
    \begin{array}{ll}
        1 & \textrm{jeśli}\ w_1w_2\cdots w_i \in L \\
        0 & \textrm{w przeciwnym wypadku}
    \end{array}
    \right.
\]

Dla języków $L \subseteq \Sigma^*$ i $K \subseteq \{0, 1\}^*$, język $K^{(L)} \subseteq \Sigma^*$ definujemy tak:
\[
    K^{(L)} = \{ w\ |\ w^{(L)} \in K\}
\]
Przykładowo, jeżeli $K$ to język wszystkich słów, w których co najmniej raz występuje litera $1$, to $K^{(L)}$ to język tych słów, w których co najmniej jeden prefiks należy do L.

Udowodnij, że jeśli $L \in \BPP$ i $K \in \BPP$ to $K^{(L)} \in \BPP$.


\subsubsection*{Rozwiązanie}

$p$ - prawdopodobieństwo błędu któregoś z automatów - języka L lub K.?

$P(\text{dobrego wyniku}) \geq (1 - p)^n(1-p) = (1-p)^{n+1}$

$P(\text{złego wyniku}) \leq 1 - (1 - p)^{n+1} < p(n+1)$

nierówność Bernouliego: $(1 + x)^n \leq 1 + nx$

$p(n+1) < \frac{1}{3}$

$p < \frac{1}{3n + 3}$

amplifikacja i już

\section{Egzamin 2015/2016 poprawkowy}

\subsection{Problem 1}
Powiemy, że formuła jest w postaci (2, 3)-CNF, jeśli jest koniunkcją klauzul, z których każda zawiera 2 lub 3 literały. O wartościowaniu powiemy, że 2-spełnia taką formułę, jeśli każda klauzula złożona dwóch literałów jest spełniona w zwykłym sensie, natomiast w każdej klauzuli złożonej z trzech literałów, dokładnie dwa są prawdziwe. Na przykład formuła $(x \lor y) \land (y \lor z) \land (z \lor x) \land (x \lor y \lor z)$ jest spełniona przez wartościowanie $x \mapsto 1$, $y \mapsto 1$, $z \mapsto 0$, a formuła $(x \lor y \lor z) \land (x \lor y \lor z)$ nie jest spełniona przez żadne wartościowanie.

Udowodnić, że następujący problem jest $\NP$-zupełny: Dla danej formuły $w$ postaci (2,3)-CNF, rozstrzygnąć, czy istnieje wartościowanie, które 2-spełnia tę formułę.

\subsubsection*{Rozwiązanie}

Redukcja z 3-kolorwania. Dostajemy graf i dla każdego wierzchołka $v$ robimy 3 zmienne $v_1$, $v_2$, $v_3$, w następujący sposób:

\[
    v_i = \left\{ \begin{array}{ll} 0 & jeśli\ col(v) = i \\ 1 & wpp \end{array} \right.
\]

I dla każdego wierzchołka tworzymy takie klauzule: $(v_1 \veebar v_2 \veebar v_3)$ ($\veebar$ to oznaczenie na ten specjalny lub z zadania, gdzie dokładnie dwie są prawdziwe).

Dla każdej krawędzi od $a$ do $b$ robimy klauzule: $(a_1 \lor b_1) \land (a_2 \lor b_2) \land (a_3 \lor b_3)$.

\subsection{Problem 2}
Dla danego skończonego ciągu liczb naturalnych (nieposortowanego) i liczby $i$, mamy wskazać liczbę, która po posortowaniu znalazłaby się na $i$-tym miejscu. Udowodnić, że istnieje deterministyczna maszyna Turinga, która rozwiązuje ten problem, pracując w roboczej pamięci logarytmicznej. Zakładamy, że słowo wejściowe jest w postaci $w_1\#\cdots\#w_k\#i$, gdzie $w_1, \cdots, w_k$ oraz $i$ są liczbami zapisanymi binarnie. Jeżeli format jest niewłaściwy (np. $i > k$), maszyna sygnalizuje błąd.

\textbf{Wskazówka}. Rozwiązać najpierw szczególny przypadek, kiedy w ciągu nie ma powtórzeń. (Rozwiązania ograniczone do tego przypadku też będą punktowane.)

\subsubsection*{Rozwiązanie}

Po kolei szukamy kolejnych liczb w kolejności rosnącej, robimy to $i$ razy. Z powtórzeniami: po znalezieniu kolejnej liczby, ponownie przechodzimy przez te liczby i zliczamy liczbę jej wystąpień. O tyle zwiększami licznik $i$.

\subsection{Problem 3}
Zakładamy, że lewy nawias jest reprezentowany przez $0$, a prawy przez $1$. Udowodnić, że język poprawnych wyrażeń nawiasowych jest w klasie $\AC^1$, tzn. może być rozpoznany przez ciąg obwodów o rozmiarze wielomianowym i głębokości logarytmicznej, przy dowolnym stopniu wejścia.

\textit{Uwaga}. Można udowodnić silniejsze stwierdzenie, że interesujący nas język jest w klasie $\NC^1$.

\subsubsection*{Rozwiązanie}

Wiadomo, że $\L \in \AC^1$, więc wystarczy napisać prostą pętlę z licznikiem otwartych nawiasów.


\textbf{Inny sposób}:

Sprawdzamy parami, logarytmiczna wysokość:




\tikzset{every picture/.style={line width=0.75pt}} %set default line width to 0.75pt        

\begin{tikzpicture}[x=0.75pt,y=0.75pt,yscale=-1,xscale=1]
%uncomment if require: \path (0,188); %set diagram left start at 0, and has height of 188

%Straight Lines [id:da3593097652792149] 
\draw [color={rgb, 255:red, 173; green, 173; blue, 173 }  ,draw opacity=1 ]   (2,118) -- (40.21,98.89) ;
\draw [shift={(42,98)}, rotate = 513.4300000000001] [color={rgb, 255:red, 173; green, 173; blue, 173 }  ,draw opacity=1 ][line width=0.75]    (10.93,-3.29) .. controls (6.95,-1.4) and (3.31,-0.3) .. (0,0) .. controls (3.31,0.3) and (6.95,1.4) .. (10.93,3.29)   ;

%Straight Lines [id:da7518110952581214] 
\draw [color={rgb, 255:red, 173; green, 173; blue, 173 }  ,draw opacity=1 ]   (82,118) -- (43.79,98.89) ;
\draw [shift={(42,98)}, rotate = 386.57] [color={rgb, 255:red, 173; green, 173; blue, 173 }  ,draw opacity=1 ][line width=0.75]    (10.93,-3.29) .. controls (6.95,-1.4) and (3.31,-0.3) .. (0,0) .. controls (3.31,0.3) and (6.95,1.4) .. (10.93,3.29)   ;

%Straight Lines [id:da3989726935629676] 
\draw [color={rgb, 255:red, 173; green, 173; blue, 173 }  ,draw opacity=1 ]   (82,118) -- (120.21,98.89) ;
\draw [shift={(122,98)}, rotate = 513.4300000000001] [color={rgb, 255:red, 173; green, 173; blue, 173 }  ,draw opacity=1 ][line width=0.75]    (10.93,-3.29) .. controls (6.95,-1.4) and (3.31,-0.3) .. (0,0) .. controls (3.31,0.3) and (6.95,1.4) .. (10.93,3.29)   ;

%Straight Lines [id:da28524307754309364] 
\draw [color={rgb, 255:red, 173; green, 173; blue, 173 }  ,draw opacity=1 ]   (162,118) -- (123.79,98.89) ;
\draw [shift={(122,98)}, rotate = 386.57] [color={rgb, 255:red, 173; green, 173; blue, 173 }  ,draw opacity=1 ][line width=0.75]    (10.93,-3.29) .. controls (6.95,-1.4) and (3.31,-0.3) .. (0,0) .. controls (3.31,0.3) and (6.95,1.4) .. (10.93,3.29)   ;

%Straight Lines [id:da5547934189964561] 
\draw [color={rgb, 255:red, 173; green, 173; blue, 173 }  ,draw opacity=1 ]   (162,118) -- (200.21,98.89) ;
\draw [shift={(202,98)}, rotate = 513.4300000000001] [color={rgb, 255:red, 173; green, 173; blue, 173 }  ,draw opacity=1 ][line width=0.75]    (10.93,-3.29) .. controls (6.95,-1.4) and (3.31,-0.3) .. (0,0) .. controls (3.31,0.3) and (6.95,1.4) .. (10.93,3.29)   ;

%Straight Lines [id:da0529524088550164] 
\draw [color={rgb, 255:red, 173; green, 173; blue, 173 }  ,draw opacity=1 ]   (242,118) -- (203.79,98.89) ;
\draw [shift={(202,98)}, rotate = 386.57] [color={rgb, 255:red, 173; green, 173; blue, 173 }  ,draw opacity=1 ][line width=0.75]    (10.93,-3.29) .. controls (6.95,-1.4) and (3.31,-0.3) .. (0,0) .. controls (3.31,0.3) and (6.95,1.4) .. (10.93,3.29)   ;

%Straight Lines [id:da9834518324618562] 
\draw [color={rgb, 255:red, 173; green, 173; blue, 173 }  ,draw opacity=1 ]   (242,118) -- (280.21,98.89) ;
\draw [shift={(282,98)}, rotate = 513.4300000000001] [color={rgb, 255:red, 173; green, 173; blue, 173 }  ,draw opacity=1 ][line width=0.75]    (10.93,-3.29) .. controls (6.95,-1.4) and (3.31,-0.3) .. (0,0) .. controls (3.31,0.3) and (6.95,1.4) .. (10.93,3.29)   ;

%Straight Lines [id:da07704377717294675] 
\draw [color={rgb, 255:red, 173; green, 173; blue, 173 }  ,draw opacity=1 ]   (322,118) -- (283.79,98.89) ;
\draw [shift={(282,98)}, rotate = 386.57] [color={rgb, 255:red, 173; green, 173; blue, 173 }  ,draw opacity=1 ][line width=0.75]    (10.93,-3.29) .. controls (6.95,-1.4) and (3.31,-0.3) .. (0,0) .. controls (3.31,0.3) and (6.95,1.4) .. (10.93,3.29)   ;

%Straight Lines [id:da6769766396460951] 
\draw [color={rgb, 255:red, 173; green, 173; blue, 173 }  ,draw opacity=1 ]   (42,78) -- (80.21,58.89) ;
\draw [shift={(82,58)}, rotate = 513.4300000000001] [color={rgb, 255:red, 173; green, 173; blue, 173 }  ,draw opacity=1 ][line width=0.75]    (10.93,-3.29) .. controls (6.95,-1.4) and (3.31,-0.3) .. (0,0) .. controls (3.31,0.3) and (6.95,1.4) .. (10.93,3.29)   ;

%Straight Lines [id:da6864183925198857] 
\draw [color={rgb, 255:red, 173; green, 173; blue, 173 }  ,draw opacity=1 ]   (122,78) -- (83.79,58.89) ;
\draw [shift={(82,58)}, rotate = 386.57] [color={rgb, 255:red, 173; green, 173; blue, 173 }  ,draw opacity=1 ][line width=0.75]    (10.93,-3.29) .. controls (6.95,-1.4) and (3.31,-0.3) .. (0,0) .. controls (3.31,0.3) and (6.95,1.4) .. (10.93,3.29)   ;

%Straight Lines [id:da10745061998548511] 
\draw [color={rgb, 255:red, 173; green, 173; blue, 173 }  ,draw opacity=1 ]   (202,78) -- (240.21,58.89) ;
\draw [shift={(242,58)}, rotate = 513.4300000000001] [color={rgb, 255:red, 173; green, 173; blue, 173 }  ,draw opacity=1 ][line width=0.75]    (10.93,-3.29) .. controls (6.95,-1.4) and (3.31,-0.3) .. (0,0) .. controls (3.31,0.3) and (6.95,1.4) .. (10.93,3.29)   ;

%Straight Lines [id:da9086202440172113] 
\draw [color={rgb, 255:red, 173; green, 173; blue, 173 }  ,draw opacity=1 ]   (282,78) -- (243.79,58.89) ;
\draw [shift={(242,58)}, rotate = 386.57] [color={rgb, 255:red, 173; green, 173; blue, 173 }  ,draw opacity=1 ][line width=0.75]    (10.93,-3.29) .. controls (6.95,-1.4) and (3.31,-0.3) .. (0,0) .. controls (3.31,0.3) and (6.95,1.4) .. (10.93,3.29)   ;

%Straight Lines [id:da8406326177148788] 
\draw [color={rgb, 255:red, 173; green, 173; blue, 173 }  ,draw opacity=1 ]   (82,38) -- (160.06,18.49) ;
\draw [shift={(162,18)}, rotate = 525.96] [color={rgb, 255:red, 173; green, 173; blue, 173 }  ,draw opacity=1 ][line width=0.75]    (10.93,-3.29) .. controls (6.95,-1.4) and (3.31,-0.3) .. (0,0) .. controls (3.31,0.3) and (6.95,1.4) .. (10.93,3.29)   ;

%Straight Lines [id:da9535271285404149] 
\draw [color={rgb, 255:red, 173; green, 173; blue, 173 }  ,draw opacity=1 ]   (242,38) -- (163.94,18.49) ;
\draw [shift={(162,18)}, rotate = 374.03999999999996] [color={rgb, 255:red, 173; green, 173; blue, 173 }  ,draw opacity=1 ][line width=0.75]    (10.93,-3.29) .. controls (6.95,-1.4) and (3.31,-0.3) .. (0,0) .. controls (3.31,0.3) and (6.95,1.4) .. (10.93,3.29)   ;


% Text Node
\draw (166,156.5) node [scale=2.074] [align=left] {( \ \ \ ( \ \ \ ) \ \ \ ( \ \ \ ( \ \ \ ) \ \ \ ) \ \ \ )};
% Text Node
\draw (39.5,89) node  [align=left] {0 2};
% Text Node
\draw (119.5,89) node  [align=left] {1 1};
% Text Node
\draw (199.5,89) node  [align=left] {0 0};
% Text Node
\draw (279.5,87) node  [align=left] {2 \ 0};
% Text Node
\draw (84.5,49) node  [align=left] {0 2};
% Text Node
\draw (239.5,49) node  [align=left] {2 0};
% Text Node
\draw (164.5,9) node  [align=left] {0 0};
% Text Node
\draw (75,11) node  [align=left] {(0, 0) czyli ok →};
% Text Node
\draw (378,148) node  [align=left] {{\Huge (}};
% Text Node
\draw (376,119) node  [align=left] {x - y};
% Text Node
\draw (322,26) node [scale=0.8] [align=left] {ile otwierających\\musi być na lewo};
% Text Node
\draw (382,64) node [scale=0.8] [align=left] {ile zamykających\\musi być na prawo};
% Text Node
\draw (12,135) node  [align=left] {0-1};
% Text Node
\draw (58,135) node  [align=left] {0-1};
% Text Node
\draw (148,135) node  [align=left] {0-1};
% Text Node
\draw (192,135) node  [align=left] {0-1};
% Text Node
\draw (92,135) node  [align=left] {1-0};
% Text Node
\draw (232,135) node  [align=left] {1-0};
% Text Node
\draw (272,135) node  [align=left] {1-0};
% Text Node
\draw (322,135) node  [align=left] {1-0};
% Connection
\draw [color={rgb, 255:red, 173; green, 173; blue, 173 }  ,draw opacity=1 ]   (319.79,42) .. controls (312,76.11) and (324.24,98.88) .. (356.51,110.32) ;
\draw [shift={(358,110.84)}, rotate = 198.65] [color={rgb, 255:red, 173; green, 173; blue, 173 }  ,draw opacity=1 ][line width=0.75]    (10.93,-3.29) .. controls (6.95,-1.4) and (3.31,-0.3) .. (0,0) .. controls (3.31,0.3) and (6.95,1.4) .. (10.93,3.29)   ;

% Connection
\draw [color={rgb, 255:red, 173; green, 173; blue, 173 }  ,draw opacity=1 ]   (395.93,80) .. controls (410.99,90.27) and (410.07,99.37) .. (393.15,107.27) ;
\draw [shift={(391.53,108)}, rotate = 336.39] [color={rgb, 255:red, 173; green, 173; blue, 173 }  ,draw opacity=1 ][line width=0.75]    (10.93,-3.29) .. controls (6.95,-1.4) and (3.31,-0.3) .. (0,0) .. controls (3.31,0.3) and (6.95,1.4) .. (10.93,3.29)   ;


\end{tikzpicture}



\section{Egzamin 2016/2017}
\subsection{Problem 1}
Udowodnij, że następujący problem jest zupełny w klasie $\NL$ ze względu na redukcje w pamięci logarytmicznej: Czy dany graf skierowany jest silnie spójny? (Zakładamy, że graf jest reprezentowany przez macierz incydencji.)

\subsubsection*{Rozwiązanie}

\begin{itemize}
    \item $\NL$-trudność
    
    Chcemy rozwiązać problem reachability za pomocą naszego problemu silnej spójności. Weźmy sobie graf, dla którego chcemy sprawdzić reachability z $s$ do $t$. No i teraz robimy coś takiego: ze wszystkich wierzchołków (poza $t$) do wierzchołka $s$ dajemy krawędź i z wierzchołka $t$ do wszystkich innych wierzchołków dajemy krawędź (\textbf{także} do s). Pytamy czy jest silnie spójny. Jeśli jest silnie spójny, to znaczy że jest jakaś ścieżka z $s$ do $t$, no bo taka jest definicja silnej spójności. Czy jeśli nie będzie silnie spójny to znaczy że nie było ścieżki z $s$ do $t$, to widać, że tak. Bo nie dodaliśmy do $s$ nic wychodzącego, nie dodaliśmy do $t$ nic wchodzącego. 
    
    
    \item jest $\NL$
    
    Mamy graf i wierzchołki, silną spójność sprawdzamy za pomocą reachability, po kolei. Mamy porządek na wierzchołkach i chcemy sprawdzić czy każdy jest osiągalny z każdego. Wybierzmy po kolei wierzchołki wszystkie pary wierzchołków i sprawdzamy czy są osiągalne, tak robimy dla każdej pary. Jeśli gdzieś wierzchołki nie są osiągalne, to odrzucamy cały bieg, jeśli dobrze - sprawdzamy kolejne wierzchołki. Jeśli graf jest spójny, to będzie jeden bieg, który akceptuje wszystko (o ile jest spójny).
    
    Drugi sposób: pokażemy, że dopełnienie spójności jest w $\co\NL$ (możemy to zrobić, bo $\NL = \co\NL$). Jak pokazać, że graf jest niespójny w $\co\NL$ ($=\NL$)? Niedeterministycznie wybieramy dwa wierzchołki i sprawdzamy czy są osiągalne. Jeśli nie są osiągalne, to graf jest niespójny, koniec. 
    
\end{itemize}

Jest w $\NL$ i $\NL$-trudny, więc jest $\NL$-zupełny.

\subsection{Problem 2}
Przypuśćmy, że istnieje deterministyczny wielomianowy algorytm A, który z błędem $\frac{2}{5}$ aproksymuje prawdopodobieństwo, że dany obwód $C$ akceptuje losowe wejście. Dokładniej, dla obwodu $C(x_1, \cdots , x_n)$ algorytm oblicza liczbę wymierną A(C), taką że
\[
    |Pr(C(U_n) = 1) - A(C_n)| \leq \frac{2}{5}
\]

(\textit{note} w treści zadnia był błąd - $C(A)$ nie ma sensu)

(Gdzie zmienna losowa $U_n$ przyjmuje wartości w zbiorze ${0, 1}^n$ z jednostajnym prawdopodobieństwem.) Dowiedź, że wtedy $\P = \BPP$.

\subsubsection*{Rozwiązanie}

$A$ - algorytm wielomianowy

$C_n$ - obwód, który oblicza wartość $A(C)$ (w przybliżeniu dla ilu słów wejściowych ten obwód akceptuje)

Możemy założyć, że $r$ jest wielomianowej długości ($\exists$ wielomian $p$ taki, że $M$ czyta $\leq p(|w|)$ bitów losowych

Możemy założyć, że ten ciąg losowych bitów $r$ mamy na wejściu.

$L \in \BPP \implies \exists_M$, która działa w czasie wielomianowym

$w \in L \implies Pr_r(M\ \text{akceptuje}\ w) \geq \overbrace{\frac{19}{20}}^{\text{możemy założyć dowolną liczbę}}$ 

$w \not\in L \implies Pr_r(M\ \text{odrzuca}\ w) \leq \frac{1}{20}$


Przekształcamy maszynę na obwód, a ciąg bitów na stałe kodujemy w tej maszynie. Obwód symuluje $M$ na słowie wejściowym $w$ oraz na ciągu bitów losowych z wejśćia długości $p(|w|)$.

$\left|\underbrace{Pr_{w \in \{0, 1\}^n} (C_n\ \text{akceptuje}\ w)}_{\geq \frac{19}{20}\ \text{o ile}\ w\in L} - \underbrace{A(C_n)}_{\geq \frac{11}{20}}\right| \leq \frac{2}{5}$

gdy $w \not\in L$ - $Pr_{w \in \{0, 1\}^n} (C_n\ \text{akceptuje}\ w) \leq \frac{1}{10}$, a $A(C_n) \leq \frac{9}{20}$

\subsection{Problem 3}

Niech $M$ będzie niedeterministyczną maszyną Turinga działającą w pamięci wielomianowej. Wykaż, że istnieje deterministyczna maszyna pracująca w czasie wykładniczym (tzn. $2^{n^k}$ , dla pewnego $k$), która dla wejścia w oblicza liczbę obliczeń akceptujących maszyny M na słowie w. (Można założyć, że M nie ma obliczeń nieskończonych.)

\subsubsection*{Rozwiązanie}

\sout{Mamy bardzo dużo czasu, więc po prostu na deterministycznej symulujemy pałowo maszynę niedeterministyczną symulując jej biegi, a potem wykonując backtracking - czyli po prostu przejście po grafie, nawet bez znaczenia czy DFS czy BFS, mamy dużo miejsca, więc po prostu trzymamy stos.}

Dzięki założeniu z treści zadania, że nie ma nieskończonych obliczeń, graf konfiguracji tej maszyny nie ma cykli - jest DAGiem. Bez tego założenia to rozwiązanie nie działa.

Chcemy policzyć liczbę ścieżek akceptujących na grafie konfiguracji. Przechodzimy zatem cały graf konfiguracji (ile kosztuje jego wygenerowanie?) DFSem, ale podczas przechodzenia chcemy zliczać \textbf{ile jest ścieżek od danego wierzchołka do jakiegoś stanu akceptującego}. To już jest proste, przykład:




\tikzset{every picture/.style={line width=0.75pt}} %set default line width to 0.75pt        

\begin{tikzpicture}[x=0.75pt,y=0.75pt,yscale=-1,xscale=1]
%uncomment if require: \path (0,145); %set diagram left start at 0, and has height of 145


% Text Node
\draw  [color={rgb, 255:red, 167; green, 167; blue, 167 }  ,draw opacity=1 ][fill={rgb, 255:red, 215; green, 255; blue, 202 }  ,fill opacity=1 ]  (63.42, 104.46) circle [x radius= 17.61, y radius= 17.61]   ;
\draw (63.42,104.46) node  [align=left] {acc};
% Text Node
\draw  [color={rgb, 255:red, 167; green, 167; blue, 167 }  ,draw opacity=1 ]  (19.33, 58.21) circle [x radius= 16.8, y radius= 16.8]   ;
\draw (19.33,58.21) node  [align=left] { \ \ \ \ };
% Text Node
\draw  [color={rgb, 255:red, 167; green, 167; blue, 167 }  ,draw opacity=1 ]  (66.4, 17.92) circle [x radius= 16.8, y radius= 16.8]   ;
\draw (66.4,17.92) node  [align=left] { \ \ \ \ };
% Text Node
\draw  [color={rgb, 255:red, 167; green, 167; blue, 167 }  ,draw opacity=1 ]  (142.47, 59.7) circle [x radius= 16.8, y radius= 16.8]   ;
\draw (142.47,59.7) node  [align=left] { \ \ \ \ };
% Text Node
\draw  [color={rgb, 255:red, 167; green, 167; blue, 167 }  ,draw opacity=1 ][fill={rgb, 255:red, 215; green, 255; blue, 202 }  ,fill opacity=1 ]  (179.02, 22.4) circle [x radius= 17.61, y radius= 17.61]   ;
\draw (179.02,22.4) node  [align=left] {acc};
% Text Node
\draw  [color={rgb, 255:red, 167; green, 167; blue, 167 }  ,draw opacity=1 ]  (196.04, 104.46) circle [x radius= 16.8, y radius= 16.8]   ;
\draw (196.04,104.46) node  [align=left] { \ \ \ \ };
% Text Node
\draw  [color={rgb, 255:red, 167; green, 167; blue, 167 }  ,draw opacity=1 ]  (222.15, 59.7) circle [x radius= 16.8, y radius= 16.8]   ;
\draw (222.15,59.7) node  [align=left] { \ \ \ \ };
% Text Node
\draw  [color={rgb, 255:red, 167; green, 167; blue, 167 }  ,draw opacity=1 ]  (271.76, 124.91) circle [x radius= 16.8, y radius= 16.8]   ;
\draw (271.76,124.91) node  [align=left] { \ \ \ \ };
% Text Node
\draw  [color={rgb, 255:red, 167; green, 167; blue, 167 }  ,draw opacity=1 ]  (313.5, 71.74) circle [x radius= 16.8, y radius= 16.8]   ;
\draw (313.5,71.74) node  [align=left] { \ \ \ \ };
% Text Node
\draw (314.5,71) node [color={rgb, 255:red, 1; green, 1; blue, 111 }  ,opacity=1 ] [align=left] {6};
% Text Node
\draw (273.5,124) node [color={rgb, 255:red, 1; green, 1; blue, 111 }  ,opacity=1 ] [align=left] {6};
% Text Node
\draw (196.5,104) node [color={rgb, 255:red, 1; green, 1; blue, 111 }  ,opacity=1 ] [align=left] {6};
% Text Node
\draw (222.5,59) node [color={rgb, 255:red, 1; green, 1; blue, 111 }  ,opacity=1 ] [align=left] {3};
% Text Node
\draw (142.5,59) node [color={rgb, 255:red, 1; green, 1; blue, 111 }  ,opacity=1 ] [align=left] {3};
% Text Node
\draw (67.5,18) node [color={rgb, 255:red, 1; green, 1; blue, 111 }  ,opacity=1 ] [align=left] {1};
% Text Node
\draw (401.5,46) node [color={rgb, 255:red, 109; green, 109; blue, 109 }  ,opacity=1 ] [align=left] {{\scriptsize liczba ścieżek }\\{\scriptsize akceptujących}\\{\scriptsize startujących od}\\{\scriptsize danego wierzchołka}};
% Connection
\draw [color={rgb, 255:red, 167; green, 167; blue, 167 }  ,draw opacity=1 ]   (127.85,67.98) -- (80.49,94.79) ;
\draw [shift={(78.75,95.78)}, rotate = 330.48] [color={rgb, 255:red, 167; green, 167; blue, 167 }  ,draw opacity=1 ][line width=0.75]    (10.93,-3.29) .. controls (6.95,-1.4) and (3.31,-0.3) .. (0,0) .. controls (3.31,0.3) and (6.95,1.4) .. (10.93,3.29)   ;

% Connection
\draw [color={rgb, 255:red, 167; green, 167; blue, 167 }  ,draw opacity=1 ]   (127.74,51.61) -- (82.89,26.97) ;
\draw [shift={(81.13,26.01)}, rotate = 388.78] [color={rgb, 255:red, 167; green, 167; blue, 167 }  ,draw opacity=1 ][line width=0.75]    (10.93,-3.29) .. controls (6.95,-1.4) and (3.31,-0.3) .. (0,0) .. controls (3.31,0.3) and (6.95,1.4) .. (10.93,3.29)   ;

% Connection
\draw [color={rgb, 255:red, 167; green, 167; blue, 167 }  ,draw opacity=1 ]   (154.23,47.7) -- (165.3,36.41) ;
\draw [shift={(166.7,34.98)}, rotate = 494.42] [color={rgb, 255:red, 167; green, 167; blue, 167 }  ,draw opacity=1 ][line width=0.75]    (10.93,-3.29) .. controls (6.95,-1.4) and (3.31,-0.3) .. (0,0) .. controls (3.31,0.3) and (6.95,1.4) .. (10.93,3.29)   ;

% Connection
\draw [color={rgb, 255:red, 167; green, 167; blue, 167 }  ,draw opacity=1 ]   (53.64,28.84) -- (33.61,45.98) ;
\draw [shift={(32.09,47.28)}, rotate = 319.45] [color={rgb, 255:red, 167; green, 167; blue, 167 }  ,draw opacity=1 ][line width=0.75]    (10.93,-3.29) .. controls (6.95,-1.4) and (3.31,-0.3) .. (0,0) .. controls (3.31,0.3) and (6.95,1.4) .. (10.93,3.29)   ;

% Connection
\draw [color={rgb, 255:red, 167; green, 167; blue, 167 }  ,draw opacity=1 ]   (205.35,59.7) -- (161.27,59.7) ;
\draw [shift={(159.27,59.7)}, rotate = 360] [color={rgb, 255:red, 167; green, 167; blue, 167 }  ,draw opacity=1 ][line width=0.75]    (10.93,-3.29) .. controls (6.95,-1.4) and (3.31,-0.3) .. (0,0) .. controls (3.31,0.3) and (6.95,1.4) .. (10.93,3.29)   ;

% Connection
\draw [color={rgb, 255:red, 167; green, 167; blue, 167 }  ,draw opacity=1 ]   (183.15,93.69) -- (156.9,71.75) ;
\draw [shift={(155.36,70.47)}, rotate = 399.88] [color={rgb, 255:red, 167; green, 167; blue, 167 }  ,draw opacity=1 ][line width=0.75]    (10.93,-3.29) .. controls (6.95,-1.4) and (3.31,-0.3) .. (0,0) .. controls (3.31,0.3) and (6.95,1.4) .. (10.93,3.29)   ;

% Connection
\draw [color={rgb, 255:red, 167; green, 167; blue, 167 }  ,draw opacity=1 ]   (204.51,89.95) -- (212.68,75.94) ;
\draw [shift={(213.68,74.21)}, rotate = 480.26] [color={rgb, 255:red, 167; green, 167; blue, 167 }  ,draw opacity=1 ][line width=0.75]    (10.93,-3.29) .. controls (6.95,-1.4) and (3.31,-0.3) .. (0,0) .. controls (3.31,0.3) and (6.95,1.4) .. (10.93,3.29)   ;

% Connection
\draw [color={rgb, 255:red, 167; green, 167; blue, 167 }  ,draw opacity=1 ]   (303.12,84.95) -- (283.37,110.12) ;
\draw [shift={(282.13,111.69)}, rotate = 308.14] [color={rgb, 255:red, 167; green, 167; blue, 167 }  ,draw opacity=1 ][line width=0.75]    (10.93,-3.29) .. controls (6.95,-1.4) and (3.31,-0.3) .. (0,0) .. controls (3.31,0.3) and (6.95,1.4) .. (10.93,3.29)   ;

% Connection
\draw [color={rgb, 255:red, 167; green, 167; blue, 167 }  ,draw opacity=1 ]   (255.53,120.53) -- (214.19,109.36) ;
\draw [shift={(212.26,108.84)}, rotate = 375.11] [color={rgb, 255:red, 167; green, 167; blue, 167 }  ,draw opacity=1 ][line width=0.75]    (10.93,-3.29) .. controls (6.95,-1.4) and (3.31,-0.3) .. (0,0) .. controls (3.31,0.3) and (6.95,1.4) .. (10.93,3.29)   ;

% Connection
\draw [color={rgb, 255:red, 167; green, 167; blue, 167 }  ,draw opacity=1 ]   (65.83,34.71) -- (64.1,84.86) ;
\draw [shift={(64.03,86.86)}, rotate = 271.97] [color={rgb, 255:red, 167; green, 167; blue, 167 }  ,draw opacity=1 ][line width=0.75]    (10.93,-3.29) .. controls (6.95,-1.4) and (3.31,-0.3) .. (0,0) .. controls (3.31,0.3) and (6.95,1.4) .. (10.93,3.29)   ;

% Connection
\draw [color={rgb, 255:red, 172; green, 172; blue, 172 }  ,draw opacity=1 ]   (349,27.92) .. controls (291.08,2.84) and (277.05,13.07) .. (306.9,58.61) ;
\draw [shift={(307.82,60)}, rotate = 236.4] [color={rgb, 255:red, 172; green, 172; blue, 172 }  ,draw opacity=1 ][line width=0.75]    (10.93,-3.29) .. controls (6.95,-1.4) and (3.31,-0.3) .. (0,0) .. controls (3.31,0.3) and (6.95,1.4) .. (10.93,3.29)   ;


\end{tikzpicture}





\section{Egzamin 2016/2017 poprawkowy}
\subsection{Problem 1}
Udowodnij, że następujący problem jest zupełny w klasie $\PSPACE$ ze względu na redukcje w pamięci logarytmicznej:
Wejście: kod niedeterministycznej maszyny $M$, słowo $w$ nad alfabetem $\{0,1\}^*$, słowo postaci $1^n$. Pytanie: Czy $M$ akceptuje $w$ w pamięci $n$?

\subsubsection*{Rozwiązanie}

\begin{itemize}
    \item jest $\PSPACE$
    
    \sout{modyfikujemy maszynę: dodajemy taśmę, która jest licznikiem zajętego miejsca. Tzn przy chodzeniu poza taśmę, zwiększamy ten licznik. Rozpoznanie wychodzenia poza taśmę robimy za pomocą powiększonego alfabetu - każdy symbol występuje w wersji ,,odwiedzonej'' i ,,nieodwiedzonej''. Chcemy żeby bieg maszyny zwracał TAK gdy licznik miejsca przekroczy $n$ LUB gdy w tym biegu maszyna chce odrzucić słowo w mniej niż $n$ miejsca. Czyli rozpatrujemy problem odwrotny. Czyli maszyna zwraca TAK gdy $M$ NIE akceptuje słowa $w$ w pamięci $n$. Teraz wystarczy odwrócić wynik. Symulację przeprowadzamy niedeterministycznie, ale możemy to zrobić, bo $\PSPACE = \NSPACE$}. 
    
    \sout{Dlaczego nie możemy po prostu puścić maszyny, której bieg akceptuje słowo $w$ w pamięci $n$ (licząc pamięć tak samo?). Bo może się zdarzyć, że jeden bieg zwróci TAK (bo po prostu zaakceptowało słowo $w$ w pamięci $n$), drugi bieg zwróci NIE, bo w tym biegu przekroczyliśmy $n$, ale wtedy wynik całej maszyny to będzie TAK, bo tak działają niedeterministyczne maszyny - jeśli istnieje jakiś bieg akceptujący, to cała maszyna akceptuje. A my potrzebujemy czegoś odwrotnego - dlatego bierzemy dopełnienie problemu, a możemy to zrobić, bo  $\co\PSPACE = \PSPACE = \co\NSPACE = \NSPACE$}
    
    Jeśli zakładamy, że KAŻDY bieg musi być w pamięci $n$, to:
    
    Robimy dwa języki - \textbf{Acc}, język słów, które są akceptowane w pamięci $n$ (po prostu symulacja maszyny ze zliczaniem - jest $\NPSPACE$) oraz drugi język $\textbf{M}_{\textbf{ok}}$ - język słów, które nie muszą być akceptowane, ale ich sprawdzenie nie przekracza pamięci $n$. Ten język należy do klasy $\co\NPSPACE$. Dzięki temu, że $\PSPACE = \NPSPACE = \co\NPSPACE$ i te klasy są zamknięte na dopełnienie, zatem przecięcie $\textbf{Acc} \cap \textbf{M}_{\textbf{ok}}$ jest w $\PSPACE$ (a przecięcie to te słowa, które mieszczą się w pamięci $n$).
    
    jeśli zakładamy, że TYLKO bieg akceptujący musi spełniać warunek pamięci to:
    
    z douczek: niedeterministycznie symulujemy maszynę i wykrywamy zapętlenia (jeśli zrobi więcej niż wykładniczo wiele kroków (tzn. liczba konfiguracji), to kończymy symulację).
    
    \item jest $\PSPACE$-trudny
    
    robimy redukcję z dowolnego języka z klasy $\PSPACE$, ona zajmuje $p(n)$ pamięci (to jest konkretny wielomian dla tego języka, znamy go), podczas redukcji robimy wejście: $\langle w, 1^{p(|w|)} \rangle$ i to już działa. Redukcja jest logarytmiczna, bo maszyna ma wielkość stałą, wejście też, a wypisywanie licznika wymaga tylko logarytmicznej pamięci.
    
    $w \in L \iff \text{akceptacja na maszynie} M\ \text{o wejściu} \langle w, 1^{p(|w|)} \rangle$
    
\end{itemize}

\iffalse
\subsection{Problem 2}
Język $L$ nazwiemy skrajnym, jeśli istnieje wielomian $p(n)$ taki, że dla każdego $n \in \mathbb{N}$ zachodzi 
\[
   min(|\{w \in L : |w| = n\}, |\{w \not\in L : |w| = n\}|) \leq p(n)
\]
Udowodnij, że każdy skrajny język nad alfabetem $\{0,1\}^*$ należy do (niejednorodnej) klasy $\AC^0$. Czy zachodzi stwierdzenie odwrotne, tzn. czy wszystkie języki w klasie $\AC^0$ są skrajne?
\fi


\iffalse
\subsection{Problem 3}
Klasę $\textbf{BPL}$ definiujemy podobnie jak klasę $\BPP$ z dodatkowym ograniczaniem, że pamięć robocza używana przez maszynę jest ograniczona przez $O(log\ n)$.
Dokładniej, język $L$ należy do $\textbf{BPL}$ jeśli istnieje probabilistyczna maszyna Turinga $M$, która
\begin{itemize}
    \item wykorzystuje dodatkową taśmę jednokrotnego odczytu (głowica przesuwa się tylko w prawo) zawierającą bity losowe,
    \item $M$ używa pamięci logarytmicznej,
    \item dla każdego $w \in \L$, maszyna $M$ akceptuje $w$ z prawdopodobieństwem większym lub równym $\frac{3}{4}$,
    \item dla każdego $w \not\in \L$, maszyna $M$ akceptuje $w$ z prawdopodobieństwem mniejszym niż $\frac{1}{4}$.
\end{itemize}
(Zauważmy, że długość ciągu bitów losowych nie jest ograniczona logarytmicznie.)
Udowodnij, że klasa $\textbf{BPL}$ zawarta jest w $\P$.

\fi

\section{Egzamin 2017/2018}

\subsection{Problem 1}

Let
\begin{align*}
DIST = \{(G, s, t, d)\ |\ &
    \textrm{d is the length of the shortest path}\\
    & \textrm{from s to t in directed graph G}
\}
\end{align*} 
In other words, $(G,s,t,d) \in DIST$ when there is no path from $s$ to $t$ in $G$ of length smaller than $d$, but there is such a path of length $d$. Show that $DIST$ is $\NL$-complete. (Don’t forget to show that $DIST \in \NL$. Note that because $d$ is given in binary, the working memory should be $O(log(log(d)+|G|))$.)

\subsubsection*{Rozwiązanie}

\begin{itemize}
    \item jest $\NL$:
    
    ponieważ $\NL = \co\NL$, w $\NL$ sprawdzimy istnienie ścieżki długości co najwyżej $d$ (reachability - zgadujemy wierzchołki), a za pomocą $\co\NL$ sprawdzamy, że nie istnieje ścieżka krótsza niż $d$ (unreachability).
    
    \begin{itemize}
        \item $\exists$ ścieżka z $s$ do $t$ długości $\leq d$ $\in \NL$
        \item $\neg \exists$ ścieżka z $s$ do $t$ długości $< d$ $\in \co\NL = \NL$
    \end{itemize}
    
    Bierzemy przecięcie tych dwóch języków.
    
    Trzeba jeszcze uważać na pamięć, bo $d$ jest zapisane binarnie. Jeśli $d$ jest większe niż liczba wierzchołków to odrzucamy od razu. W przeciwnym wypadku $d$ jest logarytmiczne względem rozmiaru grafu.
    
    \item $\NL$-hard
    
    redukcja z unreachability - dodajemy ścieżkę między wierzchołkami $s$ do $t$ długości $n+1$ (dłuższa niż jakakolwiek ścieżka bez cykli). Puszczamy nasz program z $d = n+1$, jeśli znajdzie tę ścieżkę, to znaczy, że w oryginalnym grafie nie było ścieżki między $s$ a $t$, więc wierzchołki są nieosiągalne. Wpp. była ścieżka od $s$ do $t$ w oryginalnym grafie, tzn. były osiąganlne.
\end{itemize}

\subsection{Problem 2}
We say that a language $L \in \{0, 1\}^*$ has $\AC^0$ witnesses if there exists a polynomial $p : \mathbb{N} \rightarrow \mathbb{N}$ and a uniform sequence of circuits of polynomial size and constant depth $(C_n)_{n \in \mathbb{N}}$, where $C_n$ has $n+p(n)$ input gates, such that for every $v \in \{0, 1\}^*$, 
\[
 (v \in L) \iff (\exists\ w \in \{0, 1\}^{p(|v|)}\ such\ that\ C_{|v|}(v, w) = 1).
\]

Prove that the class of languages that have $\AC^0$ witnesses equals $\NP$.

\subsubsection*{Rozwiązanie}

definicja naszej klasy $w_{\AC_0}$: $(v \in L) \iff (\exists\ w \in \{0, 1\}^{p(|v|)}\ such\ that\ C_{|v|}(v, w) = 1)$

definicja $\NP$: $x \in L \iff \exists u \in \{0, 1\}^*\ such\ that M(x, u) = 1$ i $M$ to deterministyczna maszyna Turinga.

\begin{itemize}
    \item $w_{AC_0} \subseteq \NP$
    
    Weźmy język $L \in w_{AC_0}$. $\forall_{x \in L} M(x, u)$, ale $AC_0 \subseteq P$, dlatego $M$ może symulować $C_{|x|}$. OK
    
    \item $\NP \subseteq w_{AC_0}$
    
    Nie możemy redukować z konkretnego języka $\NP$-zupełnego, bo redukcję należałoby wykonać na obwodzie, a raczej nikt nie wie jak to zrobić.
    
    W świadku będziemy trzymać biegi maszyny. Weźmy dowolny język z klasy $\NP$ (tzn. istnieje maszyna niedetemrinistyczna, która rozpoznaje język $L \in \NP$)
    
    Chcemy taki obwód: $C_M$ akceptuje $\langle v, w \rangle \iff$ $w$ jest poprawnym biegiem akceptującym $M$ na $v$. Jeśli taki obwód zrobimy, to już dobrze.
    
\end{itemize}

\subsection{Problem 3}
For a word $w \in \{0, 1\}^∗$, consider the following randomized process:
\begin{itemize}
    \item we randomly choose a pair of positions $a, b$, such that $1 \leq a \leq b \leq |w|\ and\ b − a \leq \frac{|w|}{2}$ (every such a pair is equally probable);
    \item we reverse all bits of won positions $i \in \{a,a+1,...,b\}$ (all 0’s are changed to 1’s, and all
1’s are changed to 0’s).
\end{itemize}
For a language $L \subseteq \{0,1\}^*$, let
\[
 robust(L) = \left\{ w \in L | \textrm{Prob(the process applied to w gives a word in L)} > \cfrac{3}{4} \right\}
\]
Show that if $L \in \RP$, then $robust(L) \in \RP$.



\subsubsection*{Rozwiązanie}

Przypomnijmy definicję $\RP$:

\begin{itemize}
    \item $x \in L \iff \mathbb{P}(M(x) = L(x)) > \frac{1}{2}$
    \item $x \not\in L \iff \mathbb{P}(M(x) = L(x)) = 1$
\end{itemize}

$f(w, a, b)$ - funkcja przekształcająca zgodnie z naszym randomizowanym procesem

$robust(L) = \{w \in L : \mathbb{P}(f(w) \in L) > \frac{3}{4}\}$

to prawdopodobieństwo to po prostu stosunek: $\cfrac{|\text{słowa wygenerowane przez zamianę bitów, które są w języku L}|}{|\text{wszystkie słowa wygenerowane przez zamianę bitów}|}$.

$\forall_{a, b} w'$ $\rightarrow$ $n^2$

$w' \in L \rightarrow$ counter++

$w' \in L \land M(w)\ \text{powie, ze}\ \text{w nie nalezy}$

$\RP$ istnieje maszyna randomizowana, która zawsze odrzuci słowo gdy słowo nie należy do języka, ale zaakeptuje słowo z p-stwem $\frac{1}{2}$  (z amplifikacji może wyjść nawet $\frac{3}{4}$ ? a nawet możemy zrobić $> 1 - \frac{1}{5n^2}$, a nawet $> 1 - \frac{1}{2^k}$) 

Mamy zatem algorytm, który prawie zawsze działa dobrze. Jest wielomianowo wiele słów, próbujemy każde $a$, $b$.

$\forall_{a, b}$ sprawdzamy czy $M$ zaakceptuje $zamiana(w, a, b)$. Akceptujemy jeśi dla $\geq \frac{3}{4}$ par $M$ zaakceptowała. Dlaczego to jest dobrze? Widać, że jeśli $M$ zawsze odpowiadała poprawnie, to jest ok. Jeśli $w \not\in roboust(L)$, to dzięki temu, że $M$ się myli w dobrą stronę, to prawdopodobieństwo jest 

\begin{itemize}
    \item jeśli $w \not\in robust(L) \implies$ odrzuca w
    \item jeśli $w \in robust(L)$:
    \begin{itemize}
        \item jeśli $M$ zawsze odpowiada poprawnie - OK
        \item $\forall$ wykonań $M$ spośród $\leq n^2$ wykonań, prawdopodobieństwo błędu $\leq \frac{1}{5n^2} \quad\quad n^2\cdot\frac{1}{5n^2} \leq \frac{1}{5}$
    \end{itemize}
\end{itemize}

$A_i$ - zdarzenie, że $M$ się pomyliła w $i$-tym wykonaniu

$P(A_1 \cup \cdots A_m) \leq P(A_1) + \cdots + P(A_m) \leq \underbrace{n^2}_{m\text{ - liczba wywołań}} \cdot \frac{1}{5n^2}$

\end{document}
