\subsection{Problem 1}
Udowodnij, że następujący problem jest zupełny w klasie $\NL$ ze względu na redukcje w pamięci logarytmicznej: Czy dany graf skierowany jest silnie spójny? (Zakładamy, że graf jest reprezentowany przez macierz incydencji.)

\subsubsection*{Rozwiązanie}

\begin{itemize}
    \item $\NL$-trudność
    
    Chcemy rozwiązać problem reachability za pomocą naszego problemu silnej spójności. Weźmy sobie graf, dla którego chcemy sprawdzić reachability z $s$ do $t$. No i teraz robimy coś takiego: ze wszystkich wierzchołków (poza $t$) do wierzchołka $s$ dajemy krawędź i z wierzchołka $t$ do wszystkich innych wierzchołków dajemy krawędź (\textbf{także} do s). Pytamy czy jest silnie spójny. Jeśli jest silnie spójny, to znaczy że jest jakaś ścieżka z $s$ do $t$, no bo taka jest definicja silnej spójności. Czy jeśli nie będzie silnie spójny to znaczy że nie było ścieżki z $s$ do $t$, to widać, że tak. Bo nie dodaliśmy do $s$ nic wychodzącego, nie dodaliśmy do $t$ nic wchodzącego. 
    
    
    \item jest $\NL$
    
    Mamy graf i wierzchołki, silną spójność sprawdzamy za pomocą reachability, po kolei. Mamy porządek na wierzchołkach i chcemy sprawdzić czy każdy jest osiągalny z każdego. Wybierzmy po kolei wierzchołki wszystkie pary wierzchołków i sprawdzamy czy są osiągalne, tak robimy dla każdej pary. Jeśli gdzieś wierzchołki nie są osiągalne, to odrzucamy cały bieg, jeśli dobrze - sprawdzamy kolejne wierzchołki. Jeśli graf jest spójny, to będzie jeden bieg, który akceptuje wszystko (o ile jest spójny).
    
    Drugi sposób: pokażemy, że dopełnienie spójności jest w $\co\NL$ (możemy to zrobić, bo $\NL = \co\NL$). Jak pokazać, że graf jest niespójny w $\co\NL$ ($=\NL$)? Niedeterministycznie wybieramy dwa wierzchołki i sprawdzamy czy są osiągalne. Jeśli nie są osiągalne, to graf jest niespójny, koniec. 
    
\end{itemize}

Jest w $\NL$ i $\NL$-trudny, więc jest $\NL$-zupełny.

\subsection{Problem 2}
Przypuśćmy, że istnieje deterministyczny wielomianowy algorytm A, który z błędem $\frac{2}{5}$ aproksymuje prawdopodobieństwo, że dany obwód $C$ akceptuje losowe wejście. Dokładniej, dla obwodu $C(x_1, \cdots , x_n)$ algorytm oblicza liczbę wymierną A(C), taką że
\[
    |Pr(C(U_n) = 1) - A(C_n)| \leq \frac{2}{5}
\]

(\textit{note} w treści zadnia był błąd - $C(A)$ nie ma sensu)

(Gdzie zmienna losowa $U_n$ przyjmuje wartości w zbiorze ${0, 1}^n$ z jednostajnym prawdopodobieństwem.) Dowiedź, że wtedy $\P = \BPP$.

\subsubsection*{Rozwiązanie}

$A$ - algorytm wielomianowy

$C_n$ - obwód, który oblicza wartość $A(C)$ (w przybliżeniu dla ilu słów wejściowych ten obwód akceptuje)

Możemy założyć, że $r$ jest wielomianowej długości ($\exists$ wielomian $p$ taki, że $M$ czyta $\leq p(|w|)$ bitów losowych

Możemy założyć, że ten ciąg losowych bitów $r$ mamy na wejściu.

$L \in \BPP \implies \exists_M$, która działa w czasie wielomianowym

$w \in L \implies Pr_r(M\ \text{akceptuje}\ w) \geq \overbrace{\frac{19}{20}}^{\text{możemy założyć dowolną liczbę}}$ 

$w \not\in L \implies Pr_r(M\ \text{odrzuca}\ w) \leq \frac{1}{20}$


Przekształcamy maszynę na obwód, a ciąg bitów na stałe kodujemy w tej maszynie. Obwód symuluje $M$ na słowie wejściowym $w$ oraz na ciągu bitów losowych z wejśćia długości $p(|w|)$.

$\left|\underbrace{Pr_{w \in \{0, 1\}^n} (C_n\ \text{akceptuje}\ w)}_{\geq \frac{19}{20}\ \text{o ile}\ w\in L} - \underbrace{A(C_n)}_{\geq \frac{11}{20}}\right| \leq \frac{2}{5}$

gdy $w \not\in L$ - $Pr_{w \in \{0, 1\}^n} (C_n\ \text{akceptuje}\ w) \leq \frac{1}{10}$, a $A(C_n) \leq \frac{9}{20}$

\subsection{Problem 3}

Niech $M$ będzie niedeterministyczną maszyną Turinga działającą w pamięci wielomianowej. Wykaż, że istnieje deterministyczna maszyna pracująca w czasie wykładniczym (tzn. $2^{n^k}$ , dla pewnego $k$), która dla wejścia w oblicza liczbę obliczeń akceptujących maszyny M na słowie w. (Można założyć, że M nie ma obliczeń nieskończonych.)

\subsubsection*{Rozwiązanie}

\sout{Mamy bardzo dużo czasu, więc po prostu na deterministycznej symulujemy pałowo maszynę niedeterministyczną symulując jej biegi, a potem wykonując backtracking - czyli po prostu przejście po grafie, nawet bez znaczenia czy DFS czy BFS, mamy dużo miejsca, więc po prostu trzymamy stos.}

Dzięki założeniu z treści zadania, że nie ma nieskończonych obliczeń, graf konfiguracji tej maszyny nie ma cykli - jest DAGiem. Bez tego założenia to rozwiązanie nie działa.

Chcemy policzyć liczbę ścieżek akceptujących na grafie konfiguracji. Przechodzimy zatem cały graf konfiguracji (ile kosztuje jego wygenerowanie?) DFSem, ale podczas przechodzenia chcemy zliczać \textbf{ile jest ścieżek od danego wierzchołka do jakiegoś stanu akceptującego}. To już jest proste, przykład:




\tikzset{every picture/.style={line width=0.75pt}} %set default line width to 0.75pt        

\begin{tikzpicture}[x=0.75pt,y=0.75pt,yscale=-1,xscale=1]
%uncomment if require: \path (0,145); %set diagram left start at 0, and has height of 145


% Text Node
\draw  [color={rgb, 255:red, 167; green, 167; blue, 167 }  ,draw opacity=1 ][fill={rgb, 255:red, 215; green, 255; blue, 202 }  ,fill opacity=1 ]  (63.42, 104.46) circle [x radius= 17.61, y radius= 17.61]   ;
\draw (63.42,104.46) node  [align=left] {acc};
% Text Node
\draw  [color={rgb, 255:red, 167; green, 167; blue, 167 }  ,draw opacity=1 ]  (19.33, 58.21) circle [x radius= 16.8, y radius= 16.8]   ;
\draw (19.33,58.21) node  [align=left] { \ \ \ \ };
% Text Node
\draw  [color={rgb, 255:red, 167; green, 167; blue, 167 }  ,draw opacity=1 ]  (66.4, 17.92) circle [x radius= 16.8, y radius= 16.8]   ;
\draw (66.4,17.92) node  [align=left] { \ \ \ \ };
% Text Node
\draw  [color={rgb, 255:red, 167; green, 167; blue, 167 }  ,draw opacity=1 ]  (142.47, 59.7) circle [x radius= 16.8, y radius= 16.8]   ;
\draw (142.47,59.7) node  [align=left] { \ \ \ \ };
% Text Node
\draw  [color={rgb, 255:red, 167; green, 167; blue, 167 }  ,draw opacity=1 ][fill={rgb, 255:red, 215; green, 255; blue, 202 }  ,fill opacity=1 ]  (179.02, 22.4) circle [x radius= 17.61, y radius= 17.61]   ;
\draw (179.02,22.4) node  [align=left] {acc};
% Text Node
\draw  [color={rgb, 255:red, 167; green, 167; blue, 167 }  ,draw opacity=1 ]  (196.04, 104.46) circle [x radius= 16.8, y radius= 16.8]   ;
\draw (196.04,104.46) node  [align=left] { \ \ \ \ };
% Text Node
\draw  [color={rgb, 255:red, 167; green, 167; blue, 167 }  ,draw opacity=1 ]  (222.15, 59.7) circle [x radius= 16.8, y radius= 16.8]   ;
\draw (222.15,59.7) node  [align=left] { \ \ \ \ };
% Text Node
\draw  [color={rgb, 255:red, 167; green, 167; blue, 167 }  ,draw opacity=1 ]  (271.76, 124.91) circle [x radius= 16.8, y radius= 16.8]   ;
\draw (271.76,124.91) node  [align=left] { \ \ \ \ };
% Text Node
\draw  [color={rgb, 255:red, 167; green, 167; blue, 167 }  ,draw opacity=1 ]  (313.5, 71.74) circle [x radius= 16.8, y radius= 16.8]   ;
\draw (313.5,71.74) node  [align=left] { \ \ \ \ };
% Text Node
\draw (314.5,71) node [color={rgb, 255:red, 1; green, 1; blue, 111 }  ,opacity=1 ] [align=left] {6};
% Text Node
\draw (273.5,124) node [color={rgb, 255:red, 1; green, 1; blue, 111 }  ,opacity=1 ] [align=left] {6};
% Text Node
\draw (196.5,104) node [color={rgb, 255:red, 1; green, 1; blue, 111 }  ,opacity=1 ] [align=left] {6};
% Text Node
\draw (222.5,59) node [color={rgb, 255:red, 1; green, 1; blue, 111 }  ,opacity=1 ] [align=left] {3};
% Text Node
\draw (142.5,59) node [color={rgb, 255:red, 1; green, 1; blue, 111 }  ,opacity=1 ] [align=left] {3};
% Text Node
\draw (67.5,18) node [color={rgb, 255:red, 1; green, 1; blue, 111 }  ,opacity=1 ] [align=left] {1};
% Text Node
\draw (401.5,46) node [color={rgb, 255:red, 109; green, 109; blue, 109 }  ,opacity=1 ] [align=left] {{\scriptsize liczba ścieżek }\\{\scriptsize akceptujących}\\{\scriptsize startujących od}\\{\scriptsize danego wierzchołka}};
% Connection
\draw [color={rgb, 255:red, 167; green, 167; blue, 167 }  ,draw opacity=1 ]   (127.85,67.98) -- (80.49,94.79) ;
\draw [shift={(78.75,95.78)}, rotate = 330.48] [color={rgb, 255:red, 167; green, 167; blue, 167 }  ,draw opacity=1 ][line width=0.75]    (10.93,-3.29) .. controls (6.95,-1.4) and (3.31,-0.3) .. (0,0) .. controls (3.31,0.3) and (6.95,1.4) .. (10.93,3.29)   ;

% Connection
\draw [color={rgb, 255:red, 167; green, 167; blue, 167 }  ,draw opacity=1 ]   (127.74,51.61) -- (82.89,26.97) ;
\draw [shift={(81.13,26.01)}, rotate = 388.78] [color={rgb, 255:red, 167; green, 167; blue, 167 }  ,draw opacity=1 ][line width=0.75]    (10.93,-3.29) .. controls (6.95,-1.4) and (3.31,-0.3) .. (0,0) .. controls (3.31,0.3) and (6.95,1.4) .. (10.93,3.29)   ;

% Connection
\draw [color={rgb, 255:red, 167; green, 167; blue, 167 }  ,draw opacity=1 ]   (154.23,47.7) -- (165.3,36.41) ;
\draw [shift={(166.7,34.98)}, rotate = 494.42] [color={rgb, 255:red, 167; green, 167; blue, 167 }  ,draw opacity=1 ][line width=0.75]    (10.93,-3.29) .. controls (6.95,-1.4) and (3.31,-0.3) .. (0,0) .. controls (3.31,0.3) and (6.95,1.4) .. (10.93,3.29)   ;

% Connection
\draw [color={rgb, 255:red, 167; green, 167; blue, 167 }  ,draw opacity=1 ]   (53.64,28.84) -- (33.61,45.98) ;
\draw [shift={(32.09,47.28)}, rotate = 319.45] [color={rgb, 255:red, 167; green, 167; blue, 167 }  ,draw opacity=1 ][line width=0.75]    (10.93,-3.29) .. controls (6.95,-1.4) and (3.31,-0.3) .. (0,0) .. controls (3.31,0.3) and (6.95,1.4) .. (10.93,3.29)   ;

% Connection
\draw [color={rgb, 255:red, 167; green, 167; blue, 167 }  ,draw opacity=1 ]   (205.35,59.7) -- (161.27,59.7) ;
\draw [shift={(159.27,59.7)}, rotate = 360] [color={rgb, 255:red, 167; green, 167; blue, 167 }  ,draw opacity=1 ][line width=0.75]    (10.93,-3.29) .. controls (6.95,-1.4) and (3.31,-0.3) .. (0,0) .. controls (3.31,0.3) and (6.95,1.4) .. (10.93,3.29)   ;

% Connection
\draw [color={rgb, 255:red, 167; green, 167; blue, 167 }  ,draw opacity=1 ]   (183.15,93.69) -- (156.9,71.75) ;
\draw [shift={(155.36,70.47)}, rotate = 399.88] [color={rgb, 255:red, 167; green, 167; blue, 167 }  ,draw opacity=1 ][line width=0.75]    (10.93,-3.29) .. controls (6.95,-1.4) and (3.31,-0.3) .. (0,0) .. controls (3.31,0.3) and (6.95,1.4) .. (10.93,3.29)   ;

% Connection
\draw [color={rgb, 255:red, 167; green, 167; blue, 167 }  ,draw opacity=1 ]   (204.51,89.95) -- (212.68,75.94) ;
\draw [shift={(213.68,74.21)}, rotate = 480.26] [color={rgb, 255:red, 167; green, 167; blue, 167 }  ,draw opacity=1 ][line width=0.75]    (10.93,-3.29) .. controls (6.95,-1.4) and (3.31,-0.3) .. (0,0) .. controls (3.31,0.3) and (6.95,1.4) .. (10.93,3.29)   ;

% Connection
\draw [color={rgb, 255:red, 167; green, 167; blue, 167 }  ,draw opacity=1 ]   (303.12,84.95) -- (283.37,110.12) ;
\draw [shift={(282.13,111.69)}, rotate = 308.14] [color={rgb, 255:red, 167; green, 167; blue, 167 }  ,draw opacity=1 ][line width=0.75]    (10.93,-3.29) .. controls (6.95,-1.4) and (3.31,-0.3) .. (0,0) .. controls (3.31,0.3) and (6.95,1.4) .. (10.93,3.29)   ;

% Connection
\draw [color={rgb, 255:red, 167; green, 167; blue, 167 }  ,draw opacity=1 ]   (255.53,120.53) -- (214.19,109.36) ;
\draw [shift={(212.26,108.84)}, rotate = 375.11] [color={rgb, 255:red, 167; green, 167; blue, 167 }  ,draw opacity=1 ][line width=0.75]    (10.93,-3.29) .. controls (6.95,-1.4) and (3.31,-0.3) .. (0,0) .. controls (3.31,0.3) and (6.95,1.4) .. (10.93,3.29)   ;

% Connection
\draw [color={rgb, 255:red, 167; green, 167; blue, 167 }  ,draw opacity=1 ]   (65.83,34.71) -- (64.1,84.86) ;
\draw [shift={(64.03,86.86)}, rotate = 271.97] [color={rgb, 255:red, 167; green, 167; blue, 167 }  ,draw opacity=1 ][line width=0.75]    (10.93,-3.29) .. controls (6.95,-1.4) and (3.31,-0.3) .. (0,0) .. controls (3.31,0.3) and (6.95,1.4) .. (10.93,3.29)   ;

% Connection
\draw [color={rgb, 255:red, 172; green, 172; blue, 172 }  ,draw opacity=1 ]   (349,27.92) .. controls (291.08,2.84) and (277.05,13.07) .. (306.9,58.61) ;
\draw [shift={(307.82,60)}, rotate = 236.4] [color={rgb, 255:red, 172; green, 172; blue, 172 }  ,draw opacity=1 ][line width=0.75]    (10.93,-3.29) .. controls (6.95,-1.4) and (3.31,-0.3) .. (0,0) .. controls (3.31,0.3) and (6.95,1.4) .. (10.93,3.29)   ;


\end{tikzpicture}





\section{Egzamin 2016/2017 poprawkowy}
\subsection{Problem 1}
Udowodnij, że następujący problem jest zupełny w klasie $\PSPACE$ ze względu na redukcje w pamięci logarytmicznej:
Wejście: kod niedeterministycznej maszyny $M$, słowo $w$ nad alfabetem $\{0,1\}^*$, słowo postaci $1^n$. Pytanie: Czy $M$ akceptuje $w$ w pamięci $n$?

\subsubsection*{Rozwiązanie}

\begin{itemize}
    \item jest $\PSPACE$
    
    \sout{modyfikujemy maszynę: dodajemy taśmę, która jest licznikiem zajętego miejsca. Tzn przy chodzeniu poza taśmę, zwiększamy ten licznik. Rozpoznanie wychodzenia poza taśmę robimy za pomocą powiększonego alfabetu - każdy symbol występuje w wersji ,,odwiedzonej'' i ,,nieodwiedzonej''. Chcemy żeby bieg maszyny zwracał TAK gdy licznik miejsca przekroczy $n$ LUB gdy w tym biegu maszyna chce odrzucić słowo w mniej niż $n$ miejsca. Czyli rozpatrujemy problem odwrotny. Czyli maszyna zwraca TAK gdy $M$ NIE akceptuje słowa $w$ w pamięci $n$. Teraz wystarczy odwrócić wynik. Symulację przeprowadzamy niedeterministycznie, ale możemy to zrobić, bo $\PSPACE = \NSPACE$}. 
    
    \sout{Dlaczego nie możemy po prostu puścić maszyny, której bieg akceptuje słowo $w$ w pamięci $n$ (licząc pamięć tak samo?). Bo może się zdarzyć, że jeden bieg zwróci TAK (bo po prostu zaakceptowało słowo $w$ w pamięci $n$), drugi bieg zwróci NIE, bo w tym biegu przekroczyliśmy $n$, ale wtedy wynik całej maszyny to będzie TAK, bo tak działają niedeterministyczne maszyny - jeśli istnieje jakiś bieg akceptujący, to cała maszyna akceptuje. A my potrzebujemy czegoś odwrotnego - dlatego bierzemy dopełnienie problemu, a możemy to zrobić, bo  $\co\PSPACE = \PSPACE = \co\NSPACE = \NSPACE$}
    
    Jeśli zakładamy, że KAŻDY bieg musi być w pamięci $n$, to:
    
    Robimy dwa języki - \textbf{Acc}, język słów, które są akceptowane w pamięci $n$ (po prostu symulacja maszyny ze zliczaniem - jest $\NPSPACE$) oraz drugi język $\textbf{M}_{\textbf{ok}}$ - język słów, które nie muszą być akceptowane, ale ich sprawdzenie nie przekracza pamięci $n$. Ten język należy do klasy $\co\NPSPACE$. Dzięki temu, że $\PSPACE = \NPSPACE = \co\NPSPACE$ i te klasy są zamknięte na dopełnienie, zatem przecięcie $\textbf{Acc} \cap \textbf{M}_{\textbf{ok}}$ jest w $\PSPACE$ (a przecięcie to te słowa, które mieszczą się w pamięci $n$).
    
    jeśli zakładamy, że TYLKO bieg akceptujący musi spełniać warunek pamięci to:
    
    z douczek: niedeterministycznie symulujemy maszynę i wykrywamy zapętlenia (jeśli zrobi więcej niż wykładniczo wiele kroków (tzn. liczba konfiguracji), to kończymy symulację).
    
    \item jest $\PSPACE$-trudny
    
    robimy redukcję z dowolnego języka z klasy $\PSPACE$, ona zajmuje $p(n)$ pamięci (to jest konkretny wielomian dla tego języka, znamy go), podczas redukcji robimy wejście: $\langle w, 1^{p(|w|)} \rangle$ i to już działa. Redukcja jest logarytmiczna, bo maszyna ma wielkość stałą, wejście też, a wypisywanie licznika wymaga tylko logarytmicznej pamięci.
    
    $w \in L \iff \text{akceptacja na maszynie} M\ \text{o wejściu} \langle w, 1^{p(|w|)} \rangle$
    
\end{itemize}

\iffalse
\subsection{Problem 2}
Język $L$ nazwiemy skrajnym, jeśli istnieje wielomian $p(n)$ taki, że dla każdego $n \in \mathbb{N}$ zachodzi 
\[
   min(|\{w \in L : |w| = n\}, |\{w \not\in L : |w| = n\}|) \leq p(n)
\]
Udowodnij, że każdy skrajny język nad alfabetem $\{0,1\}^*$ należy do (niejednorodnej) klasy $\AC^0$. Czy zachodzi stwierdzenie odwrotne, tzn. czy wszystkie języki w klasie $\AC^0$ są skrajne?
\fi


\iffalse
\subsection{Problem 3}
Klasę $\textbf{BPL}$ definiujemy podobnie jak klasę $\BPP$ z dodatkowym ograniczaniem, że pamięć robocza używana przez maszynę jest ograniczona przez $O(log\ n)$.
Dokładniej, język $L$ należy do $\textbf{BPL}$ jeśli istnieje probabilistyczna maszyna Turinga $M$, która
\begin{itemize}
    \item wykorzystuje dodatkową taśmę jednokrotnego odczytu (głowica przesuwa się tylko w prawo) zawierającą bity losowe,
    \item $M$ używa pamięci logarytmicznej,
    \item dla każdego $w \in \L$, maszyna $M$ akceptuje $w$ z prawdopodobieństwem większym lub równym $\frac{3}{4}$,
    \item dla każdego $w \not\in \L$, maszyna $M$ akceptuje $w$ z prawdopodobieństwem mniejszym niż $\frac{1}{4}$.
\end{itemize}
(Zauważmy, że długość ciągu bitów losowych nie jest ograniczona logarytmicznie.)
Udowodnij, że klasa $\textbf{BPL}$ zawarta jest w $\P$.

\fi