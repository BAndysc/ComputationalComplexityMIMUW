\subsection{Podstawowe klasy}
\begin{itemize}
    %\item $\P \subseteq \NP \subseteq \bigcup_{c > 1} \DTIME(2^{n^c})$ %
    \item $\DSPACE(f(n)) \subseteq  \bigcup_{c >0} \DTIME(n * c^{f(n)})$ 
    \item if $f(n) \geq log(n)$, then $\DSPACE(f(n)) \subseteq  \bigcup_{c >0} \DTIME(c^{f(n)})$
    %\item $\P \subsetneq \P\poly$ ($\P\poly \not\subseteq \P$)
    %\item $\NPSPACE = \PSPACE = \co\NPSPACE$, zamknięte na przecięcie %todo jest na rysunku
   
    \item $\NTIME(f(n)) \subseteq \DSPACE(f(n))$
    \item $\NSPACE(f(n)) \subseteq \cup_{c \in \mathbb{N}} \DTIME(c^{f(n) + log(n)})$
    \item (Savitch's theorem) if $f(n)=\Omega(log n)$, then $\NSPACE(f(n)) \subseteq \DSPACE(f(n)^2)$
%    \item $\NL = \co\NL$
    \item if $S(n) \geq log(n)$, then $\NSPACE(S(n))= \co\NSPACE(S(n))$
     %\item $\L \subsetneq \PSPACE$ %todo oczywiste ze space hierarchy 
    \item $\textbf{polyL} \neq \PSPACE$
    \item $\textbf{polyL} \neq \P$

\end{itemize}
\subsection{Obwody}
\begin{itemize}
    \item $\AC = \NC$, $\u\AC = \u\NC$
    \item $\AC^k \subseteq \NC^{k+1}$, $\u\AC^k \subseteq \u\NC^{k+1}$ 
%    \item $\u\NC^1 \subseteq \L \subseteq \NL \subseteq \u\AC^1 \subseteq \u\NC^2$
%    \item języki regularne $\subseteq \u\NC^1$ (ale nie $\AC^0$)
    \item $\u\AC \subseteq \P$
    \item $\u\AC^0 \neq \u\NC^1$
    \item $\u\NC \neq \PSPACE$
\end{itemize}
\subsection{Klasy probabilistyczne}
\begin{itemize}
%    \item $\P \subseteq \RP \subseteq \NP$
%    \item $\NP \subseteq \PP$
%    \item $\RP \subseteq \BPP \subseteq PP$
    \item if $\NP \subseteq \BPP$, then $\NP \subseteq \RP$
%    \item $\ZPP = \RP \cap \co\RP$
%    \item $\BPP \subseteq P\poly$
%    \item $\BPP \subseteq \PSPACE$
%    \item $\ZPP \subseteq \BPP$
%    \item $\co\RP \subseteq \BPP$
%    \item $\ZPP = \co\ZPP$
%    \item $\BPP \subseteq \textbf{BQP}  \subseteq \PSPACE$
\end{itemize}


\subsection{Polynomial hierarchy i alternating machine}
\begin{itemize}
    \item $\sum_k^p \subseteq \sum_{k+1}^p$, $\sum_k^p \subseteq \prod_{k+1}^p$, $\prod_k^p \subseteq \sum_{k+1}^p$, $\prod_k^p \subseteq \prod_{k+1}^p$
    \item if $\sum_k^p = \prod_k^p$, then $\sum_k^p = \PH$
    \item $\PH \subseteq \PSPACE$
%    \item $\AL = \P$
%    \item $\AP = \PSPACE$
    
    
    

   
    
\end{itemize}

\subsection{PCP i IP}
\begin{itemize}
    \item $PCP(poly(n), 0) = \co\RP$
    \item $PCP(0, poly(n))= \NP$
    \item $PCP(log(n), poly(n)) = \NP$
    \item $PCP(log(n), 1) = \NP$
    \item $PCP(poly(n), 1) = \EXPTIME$
    \item $PCP(1,1) = \P $
%    \item $\textbf{d}\IP = \NP$
%    \item $\IP = \PSPACE$
\end{itemize}