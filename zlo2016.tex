\subsection{Problem 1}

Opisz jak deterministycznie w pamięci logarytmicznej rozwiązać następujący problem. Dla danego ciągu wejściowego $w \in \{0, 1\}^∗$:
\begin{enumerate}
    \item sprawdź, czy $|w|$ jest postaci $n^2$ dla jakiejś liczby n,
    \item jeśli tak, to sprawdź czy $w$, rozumiane jako graf relacji binarnej na zbiorze $\{1, \cdots, n\}$, zadaje relację równoważności,
    \item jeśli tak, to wypisz (w reprezentacji binarnej) liczbę klas abstrakcji tej relacji.
\end{enumerate}

\subsubsection*{Rozwiązanie}

Intuicyjnie algorytmy w pamięci logarytmicznej korzystają ze stałej liczby wskaźników.

\begin{enumerate}
    \item Dodajemy kolejne liczby nieparzyste aż nie przekroczymy długości słowa, jak trafimy akurat w długość słowa to jest kwadratem
    
    \textbf{Prościej}: iterujemy po liczbach od $1$ do $n$, liczymy jej kwadrat (w pętli dodawanie) i sprawdzamy czy się zgadza
    \item \begin{minted}{python}
    zwrotna = True
    for x = 0 to n - 1:
        if not krawedz(x, x):
            zwrotna = False
    
    symetryczna = True
    for x = 0 to n - 1:
        for z = 0 to n - 1:
            if krawedz(x, z) and not krawedz(z, x):
                symetryczna = False
    
    przechodnia = True
    for x = 0 to n - 1:
        for y = 0 to n - 1:
            for z = 0 to n - 1:
                if krawedz(x, y) and krawedz(y, z):
                    if not krawedz(x, z):
                        przechodnia = False
                        
    rownowaznosci = zwrotna and przechodnia and symetryczna
    \end{minted}
    
    \item \begin{minted}{python}
    ileKlas = 1 # bo klasa z "0"
    for i = 0 to n - 1:
        czyNowaKlasaAbstrakcji = True
        for j = 0 to i - 1:
            if krawedz(i, j):
                czyNowaKlasaAbstrakcji = False
        if czyNowaKlasaAbstrakcji:
            ileKlas += 1
    \end{minted}
\end{enumerate}

\subsection{Problem 2}
Obwód logiczny o $2n$ bramkach wejściowych i jednej bramce wyjściowej definiuje graf skierowany o $2^n$ wierzchołkach w ten sposób, że jeśli na pewnej kombinacji wartości bramek wejściowych daje wartość $1$, to w grafie występuje krawędź z wierzchołka o numerze $K$ do wierzchołka o numerze $M$, gdzie $K$ jest binarnie reprezentowane przez ciąg wartości pierwszych $n$ bramek, a $M$ – drugich $n$ bramek.

Rozważmy następujący problem. Dane: obwód jak powyżej oraz dwie liczby binarne $K$, $M$ o $n$ bitach każda. Czy wierzchołek $M$ jest osiągalny z $K$ w grafie zdefiniowanym przez ten obwód?

Udowodnij, że ten problem jest $\PSPACE$-zupełny

\subsubsection*{Rozwiązanie}

\begin{itemize}
    \item jest $\PSPACE$
    
    niedeterministycznie zgadujemy ścieżkę między nimi i $n$ razy symulujemy ten obwód (Dla każdej pracy wierzchołków na tej ścieżce sprawdzamy czy istnieje między nimi krawędź. Zatem jest $\NPSPACE$, czyli jest $\PSPACE$. Ważne, że wygenerowany obwód jest wielomianowej wielkości.
    
    \item $\PSPACE$-trudne
\end{itemize}

\subsection{Problem 3}
Dla dowolnego języka $L \subseteq \Sigma^*$, i dla słowa $w= w_1w_2\cdots w_n \in \Sigma^n$, słowo $w^{(L)} \in \{0, 1\}^n$ jest zdefiniowane następująco:
\[
    \left(w^{(L)}\right)_i = \left\{ 
    \begin{array}{ll}
        1 & \textrm{jeśli}\ w_1w_2\cdots w_i \in L \\
        0 & \textrm{w przeciwnym wypadku}
    \end{array}
    \right.
\]

Dla języków $L \subseteq \Sigma^*$ i $K \subseteq \{0, 1\}^*$, język $K^{(L)} \subseteq \Sigma^*$ definujemy tak:
\[
    K^{(L)} = \{ w\ |\ w^{(L)} \in K\}
\]
Przykładowo, jeżeli $K$ to język wszystkich słów, w których co najmniej raz występuje litera $1$, to $K^{(L)}$ to język tych słów, w których co najmniej jeden prefiks należy do L.

Udowodnij, że jeśli $L \in \BPP$ i $K \in \BPP$ to $K^{(L)} \in \BPP$.


\subsubsection*{Rozwiązanie}

$p$ - prawdopodobieństwo błędu któregoś z automatów - języka L lub K.?

$P(\text{dobrego wyniku}) \geq (1 - p)^n(1-p) = (1-p)^{n+1}$

$P(\text{złego wyniku}) \leq 1 - (1 - p)^{n+1} < p(n+1)$

nierówność Bernouliego: $(1 + x)^n \leq 1 + nx$

$p(n+1) < \frac{1}{3}$

$p < \frac{1}{3n + 3}$

amplifikacja i już

\section{Egzamin 2015/2016 poprawkowy}

\subsection{Problem 1}
Powiemy, że formuła jest w postaci (2, 3)-CNF, jeśli jest koniunkcją klauzul, z których każda zawiera 2 lub 3 literały. O wartościowaniu powiemy, że 2-spełnia taką formułę, jeśli każda klauzula złożona dwóch literałów jest spełniona w zwykłym sensie, natomiast w każdej klauzuli złożonej z trzech literałów, dokładnie dwa są prawdziwe. Na przykład formuła $(x \lor y) \land (y \lor z) \land (z \lor x) \land (x \lor y \lor z)$ jest spełniona przez wartościowanie $x \mapsto 1$, $y \mapsto 1$, $z \mapsto 0$, a formuła $(x \lor y \lor z) \land (x \lor y \lor z)$ nie jest spełniona przez żadne wartościowanie.

Udowodnić, że następujący problem jest $\NP$-zupełny: Dla danej formuły $w$ postaci (2,3)-CNF, rozstrzygnąć, czy istnieje wartościowanie, które 2-spełnia tę formułę.

\subsubsection*{Rozwiązanie}

Redukcja z 3-kolorwania. Dostajemy graf i dla każdego wierzchołka $v$ robimy 3 zmienne $v_1$, $v_2$, $v_3$, w następujący sposób:

\[
    v_i = \left\{ \begin{array}{ll} 0 & jeśli\ col(v) = i \\ 1 & wpp \end{array} \right.
\]

I dla każdego wierzchołka tworzymy takie klauzule: $(v_1 \veebar v_2 \veebar v_3)$ ($\veebar$ to oznaczenie na ten specjalny lub z zadania, gdzie dokładnie dwie są prawdziwe).

Dla każdej krawędzi od $a$ do $b$ robimy klauzule: $(a_1 \lor b_1) \land (a_2 \lor b_2) \land (a_3 \lor b_3)$.

\subsection{Problem 2}
Dla danego skończonego ciągu liczb naturalnych (nieposortowanego) i liczby $i$, mamy wskazać liczbę, która po posortowaniu znalazłaby się na $i$-tym miejscu. Udowodnić, że istnieje deterministyczna maszyna Turinga, która rozwiązuje ten problem, pracując w roboczej pamięci logarytmicznej. Zakładamy, że słowo wejściowe jest w postaci $w_1\#\cdots\#w_k\#i$, gdzie $w_1, \cdots, w_k$ oraz $i$ są liczbami zapisanymi binarnie. Jeżeli format jest niewłaściwy (np. $i > k$), maszyna sygnalizuje błąd.

\textbf{Wskazówka}. Rozwiązać najpierw szczególny przypadek, kiedy w ciągu nie ma powtórzeń. (Rozwiązania ograniczone do tego przypadku też będą punktowane.)

\subsubsection*{Rozwiązanie}

Po kolei szukamy kolejnych liczb w kolejności rosnącej, robimy to $i$ razy. Z powtórzeniami: po znalezieniu kolejnej liczby, ponownie przechodzimy przez te liczby i zliczamy liczbę jej wystąpień. O tyle zwiększami licznik $i$.

\subsection{Problem 3}
Zakładamy, że lewy nawias jest reprezentowany przez $0$, a prawy przez $1$. Udowodnić, że język poprawnych wyrażeń nawiasowych jest w klasie $\AC^1$, tzn. może być rozpoznany przez ciąg obwodów o rozmiarze wielomianowym i głębokości logarytmicznej, przy dowolnym stopniu wejścia.

\textit{Uwaga}. Można udowodnić silniejsze stwierdzenie, że interesujący nas język jest w klasie $\NC^1$.

\subsubsection*{Rozwiązanie}

Wiadomo, że $\L \in \AC^1$, więc wystarczy napisać prostą pętlę z licznikiem otwartych nawiasów.


\textbf{Inny sposób}:

Sprawdzamy parami, logarytmiczna wysokość:




\tikzset{every picture/.style={line width=0.75pt}} %set default line width to 0.75pt        

\begin{tikzpicture}[x=0.75pt,y=0.75pt,yscale=-1,xscale=1]
%uncomment if require: \path (0,188); %set diagram left start at 0, and has height of 188

%Straight Lines [id:da3593097652792149] 
\draw [color={rgb, 255:red, 173; green, 173; blue, 173 }  ,draw opacity=1 ]   (2,118) -- (40.21,98.89) ;
\draw [shift={(42,98)}, rotate = 513.4300000000001] [color={rgb, 255:red, 173; green, 173; blue, 173 }  ,draw opacity=1 ][line width=0.75]    (10.93,-3.29) .. controls (6.95,-1.4) and (3.31,-0.3) .. (0,0) .. controls (3.31,0.3) and (6.95,1.4) .. (10.93,3.29)   ;

%Straight Lines [id:da7518110952581214] 
\draw [color={rgb, 255:red, 173; green, 173; blue, 173 }  ,draw opacity=1 ]   (82,118) -- (43.79,98.89) ;
\draw [shift={(42,98)}, rotate = 386.57] [color={rgb, 255:red, 173; green, 173; blue, 173 }  ,draw opacity=1 ][line width=0.75]    (10.93,-3.29) .. controls (6.95,-1.4) and (3.31,-0.3) .. (0,0) .. controls (3.31,0.3) and (6.95,1.4) .. (10.93,3.29)   ;

%Straight Lines [id:da3989726935629676] 
\draw [color={rgb, 255:red, 173; green, 173; blue, 173 }  ,draw opacity=1 ]   (82,118) -- (120.21,98.89) ;
\draw [shift={(122,98)}, rotate = 513.4300000000001] [color={rgb, 255:red, 173; green, 173; blue, 173 }  ,draw opacity=1 ][line width=0.75]    (10.93,-3.29) .. controls (6.95,-1.4) and (3.31,-0.3) .. (0,0) .. controls (3.31,0.3) and (6.95,1.4) .. (10.93,3.29)   ;

%Straight Lines [id:da28524307754309364] 
\draw [color={rgb, 255:red, 173; green, 173; blue, 173 }  ,draw opacity=1 ]   (162,118) -- (123.79,98.89) ;
\draw [shift={(122,98)}, rotate = 386.57] [color={rgb, 255:red, 173; green, 173; blue, 173 }  ,draw opacity=1 ][line width=0.75]    (10.93,-3.29) .. controls (6.95,-1.4) and (3.31,-0.3) .. (0,0) .. controls (3.31,0.3) and (6.95,1.4) .. (10.93,3.29)   ;

%Straight Lines [id:da5547934189964561] 
\draw [color={rgb, 255:red, 173; green, 173; blue, 173 }  ,draw opacity=1 ]   (162,118) -- (200.21,98.89) ;
\draw [shift={(202,98)}, rotate = 513.4300000000001] [color={rgb, 255:red, 173; green, 173; blue, 173 }  ,draw opacity=1 ][line width=0.75]    (10.93,-3.29) .. controls (6.95,-1.4) and (3.31,-0.3) .. (0,0) .. controls (3.31,0.3) and (6.95,1.4) .. (10.93,3.29)   ;

%Straight Lines [id:da0529524088550164] 
\draw [color={rgb, 255:red, 173; green, 173; blue, 173 }  ,draw opacity=1 ]   (242,118) -- (203.79,98.89) ;
\draw [shift={(202,98)}, rotate = 386.57] [color={rgb, 255:red, 173; green, 173; blue, 173 }  ,draw opacity=1 ][line width=0.75]    (10.93,-3.29) .. controls (6.95,-1.4) and (3.31,-0.3) .. (0,0) .. controls (3.31,0.3) and (6.95,1.4) .. (10.93,3.29)   ;

%Straight Lines [id:da9834518324618562] 
\draw [color={rgb, 255:red, 173; green, 173; blue, 173 }  ,draw opacity=1 ]   (242,118) -- (280.21,98.89) ;
\draw [shift={(282,98)}, rotate = 513.4300000000001] [color={rgb, 255:red, 173; green, 173; blue, 173 }  ,draw opacity=1 ][line width=0.75]    (10.93,-3.29) .. controls (6.95,-1.4) and (3.31,-0.3) .. (0,0) .. controls (3.31,0.3) and (6.95,1.4) .. (10.93,3.29)   ;

%Straight Lines [id:da07704377717294675] 
\draw [color={rgb, 255:red, 173; green, 173; blue, 173 }  ,draw opacity=1 ]   (322,118) -- (283.79,98.89) ;
\draw [shift={(282,98)}, rotate = 386.57] [color={rgb, 255:red, 173; green, 173; blue, 173 }  ,draw opacity=1 ][line width=0.75]    (10.93,-3.29) .. controls (6.95,-1.4) and (3.31,-0.3) .. (0,0) .. controls (3.31,0.3) and (6.95,1.4) .. (10.93,3.29)   ;

%Straight Lines [id:da6769766396460951] 
\draw [color={rgb, 255:red, 173; green, 173; blue, 173 }  ,draw opacity=1 ]   (42,78) -- (80.21,58.89) ;
\draw [shift={(82,58)}, rotate = 513.4300000000001] [color={rgb, 255:red, 173; green, 173; blue, 173 }  ,draw opacity=1 ][line width=0.75]    (10.93,-3.29) .. controls (6.95,-1.4) and (3.31,-0.3) .. (0,0) .. controls (3.31,0.3) and (6.95,1.4) .. (10.93,3.29)   ;

%Straight Lines [id:da6864183925198857] 
\draw [color={rgb, 255:red, 173; green, 173; blue, 173 }  ,draw opacity=1 ]   (122,78) -- (83.79,58.89) ;
\draw [shift={(82,58)}, rotate = 386.57] [color={rgb, 255:red, 173; green, 173; blue, 173 }  ,draw opacity=1 ][line width=0.75]    (10.93,-3.29) .. controls (6.95,-1.4) and (3.31,-0.3) .. (0,0) .. controls (3.31,0.3) and (6.95,1.4) .. (10.93,3.29)   ;

%Straight Lines [id:da10745061998548511] 
\draw [color={rgb, 255:red, 173; green, 173; blue, 173 }  ,draw opacity=1 ]   (202,78) -- (240.21,58.89) ;
\draw [shift={(242,58)}, rotate = 513.4300000000001] [color={rgb, 255:red, 173; green, 173; blue, 173 }  ,draw opacity=1 ][line width=0.75]    (10.93,-3.29) .. controls (6.95,-1.4) and (3.31,-0.3) .. (0,0) .. controls (3.31,0.3) and (6.95,1.4) .. (10.93,3.29)   ;

%Straight Lines [id:da9086202440172113] 
\draw [color={rgb, 255:red, 173; green, 173; blue, 173 }  ,draw opacity=1 ]   (282,78) -- (243.79,58.89) ;
\draw [shift={(242,58)}, rotate = 386.57] [color={rgb, 255:red, 173; green, 173; blue, 173 }  ,draw opacity=1 ][line width=0.75]    (10.93,-3.29) .. controls (6.95,-1.4) and (3.31,-0.3) .. (0,0) .. controls (3.31,0.3) and (6.95,1.4) .. (10.93,3.29)   ;

%Straight Lines [id:da8406326177148788] 
\draw [color={rgb, 255:red, 173; green, 173; blue, 173 }  ,draw opacity=1 ]   (82,38) -- (160.06,18.49) ;
\draw [shift={(162,18)}, rotate = 525.96] [color={rgb, 255:red, 173; green, 173; blue, 173 }  ,draw opacity=1 ][line width=0.75]    (10.93,-3.29) .. controls (6.95,-1.4) and (3.31,-0.3) .. (0,0) .. controls (3.31,0.3) and (6.95,1.4) .. (10.93,3.29)   ;

%Straight Lines [id:da9535271285404149] 
\draw [color={rgb, 255:red, 173; green, 173; blue, 173 }  ,draw opacity=1 ]   (242,38) -- (163.94,18.49) ;
\draw [shift={(162,18)}, rotate = 374.03999999999996] [color={rgb, 255:red, 173; green, 173; blue, 173 }  ,draw opacity=1 ][line width=0.75]    (10.93,-3.29) .. controls (6.95,-1.4) and (3.31,-0.3) .. (0,0) .. controls (3.31,0.3) and (6.95,1.4) .. (10.93,3.29)   ;


% Text Node
\draw (166,156.5) node [scale=2.074] [align=left] {( \ \ \ ( \ \ \ ) \ \ \ ( \ \ \ ( \ \ \ ) \ \ \ ) \ \ \ )};
% Text Node
\draw (39.5,89) node  [align=left] {0 2};
% Text Node
\draw (119.5,89) node  [align=left] {1 1};
% Text Node
\draw (199.5,89) node  [align=left] {0 0};
% Text Node
\draw (279.5,87) node  [align=left] {2 \ 0};
% Text Node
\draw (84.5,49) node  [align=left] {0 2};
% Text Node
\draw (239.5,49) node  [align=left] {2 0};
% Text Node
\draw (164.5,9) node  [align=left] {0 0};
% Text Node
\draw (75,11) node  [align=left] {(0, 0) czyli ok →};
% Text Node
\draw (378,148) node  [align=left] {{\Huge (}};
% Text Node
\draw (376,119) node  [align=left] {x - y};
% Text Node
\draw (322,26) node [scale=0.8] [align=left] {ile otwierających\\musi być na lewo};
% Text Node
\draw (382,64) node [scale=0.8] [align=left] {ile zamykających\\musi być na prawo};
% Text Node
\draw (12,135) node  [align=left] {0-1};
% Text Node
\draw (58,135) node  [align=left] {0-1};
% Text Node
\draw (148,135) node  [align=left] {0-1};
% Text Node
\draw (192,135) node  [align=left] {0-1};
% Text Node
\draw (92,135) node  [align=left] {1-0};
% Text Node
\draw (232,135) node  [align=left] {1-0};
% Text Node
\draw (272,135) node  [align=left] {1-0};
% Text Node
\draw (322,135) node  [align=left] {1-0};
% Connection
\draw [color={rgb, 255:red, 173; green, 173; blue, 173 }  ,draw opacity=1 ]   (319.79,42) .. controls (312,76.11) and (324.24,98.88) .. (356.51,110.32) ;
\draw [shift={(358,110.84)}, rotate = 198.65] [color={rgb, 255:red, 173; green, 173; blue, 173 }  ,draw opacity=1 ][line width=0.75]    (10.93,-3.29) .. controls (6.95,-1.4) and (3.31,-0.3) .. (0,0) .. controls (3.31,0.3) and (6.95,1.4) .. (10.93,3.29)   ;

% Connection
\draw [color={rgb, 255:red, 173; green, 173; blue, 173 }  ,draw opacity=1 ]   (395.93,80) .. controls (410.99,90.27) and (410.07,99.37) .. (393.15,107.27) ;
\draw [shift={(391.53,108)}, rotate = 336.39] [color={rgb, 255:red, 173; green, 173; blue, 173 }  ,draw opacity=1 ][line width=0.75]    (10.93,-3.29) .. controls (6.95,-1.4) and (3.31,-0.3) .. (0,0) .. controls (3.31,0.3) and (6.95,1.4) .. (10.93,3.29)   ;


\end{tikzpicture}

